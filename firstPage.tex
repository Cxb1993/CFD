
\thispagestyle{empty}

\begin{minipage}{0.5\linewidth}
	\large\textbf{CURSO DE ESCOAMENTOS\\MULTIFÁSICOS}
\end{minipage}
\begin{minipage}{0.5\linewidth}
	\flushright
	\includegraphics[height=18mm]{figs/logo_uerj_PB.png}
	\hspace{.5cm}
	\includegraphics[height=16mm]{figs/logo_ppg-em.jpg}
\end{minipage}

\hrulefill

%\begin{minipage}[b]{0.5\linewidth}
\Large \color{NavyBlue} \textbf{ESCOAMENTO EM CANAL - Modelos 1D}\\
\color{Black}\\ % Title
%\vspace{-.5cm}
\normalsize \texttt{Preparado por: Leon Lima e Gustavo Anjos}\\
\normalsize \texttt{\today}
%\end{minipage}

%\begin{minipage}[b]{0.5\linewidth}
%\flushright
%\includegraphics[height=2cm]{../../imagens/logoeletronuclear.jpg}
%\hspace{.5cm}
%\includegraphics[height=2cm]{../../imagens/logouerj.jpg}\\
%\end{minipage}

\hrulefill
\begin{center}
	\begin{minipage}{0.8\linewidth}
		\footnotesize{\textbf{Resumo.} Este texto apresenta os pontos
		principais da modelagem matemática de escoamento incompressível,
		desenvolvido, em regimes laminar e turbulento, permanente e
		transiente, em canal aberto. Conceitos básicos de turbulência
		em canais são apresentados. Além disso, a discretização das
		equações de escoamento em canal 1D utilizando os métodos de
		diferenças finitas e de elementos finitos é apresentada para os
		perfis laminar e turbulento. Ao longo do texto, o estudante
		poderá também se familizarizar com a modelagem de escoamentos
		particulados. Exercícios são propostos no final do texto.}
	\end{minipage}
\end{center}

%\begin{multicols}{2}
{\footnotesize\tableofcontents}
\hrulefill\\
\onehalfspacing
\vspace{10mm}

