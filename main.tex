\documentclass[a4paper,portuguese,10pt]{article}

\usepackage{setspace}
\usepackage[hang,footnotesize]{caption}
\usepackage[utf8]{inputenc}
\usepackage{graphicx}
%\usepackage{hyperref}
\usepackage[colorlinks=true]{hyperref}
\hypersetup{
  linkcolor=black,
  citecolor=blue,
}
%\usepackage{setspace}
%\usepackage{abntcite}
\usepackage{amsfonts}
\usepackage{amsmath}
\usepackage{nonfloat}
\usepackage{semtrans}
\usepackage[margin=2cm]{geometry}
\usepackage[portuguese]{babel}
\usepackage[fixlanguage]{babelbib}
\selectbiblanguage{portuguese}
\usepackage[svgnames]{xcolor} % Specify colors by their 'svgnames'; list of all colors available: http://www.latextemplates.com/svgnames-colors
\usepackage{titlesec}
\usepackage[numbers]{natbib}
\usepackage{nomencl}
\usepackage{ifthen}
\usepackage{multicol} % use of multiple columns
\usepackage{fancyhdr}
\usepackage{tikz}
\usetikzlibrary{shapes,shadows,arrows}
%\usepackage{draftwatermark}
%\SetWatermarkText{ANNEX 2}
%\SetWatermarkScale{3}
%\SetWatermarkColor[rgb]{red!60}
\columnsep=8mm
%\columnseprule=1pt

\newcommand{\p}{\parallel}
\newcommand{\m}{\mid}
\newcommand{\del}{\bigtriangleup}
\newcommand{\lb}{\linebreak}
\newcommand{\nl}{\newline}
%\renewcommand{\div}{{\,\rm div}\,}
\renewcommand{\div}{{\,\nabla\cdot}\,}
%\newcommand{\grad}{\,\rm {grad}\,}
\newcommand{\grad}{\,\rm \nabla\,}
\renewcommand{\D}{\partial}
\newcommand{\RR}{\mathbb{R}}
\renewcommand{\vec}{\mathbf}
\newcommand{\esp}{\text{\hspace{2mm}}}
\newcommand{\dgs}{\textordmasculine}
\renewcommand{\max}{\operatorname{max}}
\newcommand{\Ra}{\operatorname{Ra}}
\renewcommand{\Re}{\operatorname{Re}}
\newcommand{\Gr}{\operatorname{Gr}}
\newcommand{\St}{\operatorname{St}}
\renewcommand{\Pr}{\operatorname{Pr}}
\newcommand{\CFL}{\operatorname{CFL}}
\newcommand{\dt}{\Delta t}
\newcommand{\dy}{\Delta y}

%\setlength{\hoffset}{-1cm}
%\setlength{\textwidth}{17cm}
%\setlength{\parskip}{1cm plus4mm minus3mm}
\setlength{\parskip}{2mm}
\setlength{\parindent}{0mm} % Default is 15pt.
\titleformat*{\section}{\normalsize\bfseries}
\titleformat*{\subsection}{\normalsize\bfseries}
\titleformat*{\subsubsection}{\normalsize}

\renewcommand{\refname}{\normalsize{References}}
\renewcommand{\nomname}{List of Symbols}

\renewcommand{\nomgroup}[1]{%
\ifthenelse{\equal{#1}{R}}{\item[\textbf{Roman letters}]}{%
\ifthenelse{\equal{#1}{G}}{\item[Greek letters]}{}}}

\pagestyle{fancy}
\lhead{}
\rhead{\small{ESCOAMENTO EM CANAL - Modelos 1D}}


%%%%%%%%%%%%%%%%%%%%%%%%%%%%%%%%%%%%%%%%%%%%%%%%%%%%%%%%%%%%%%%%%%%%%%%%%%%%%%%%%%
\begin{document}

\thispagestyle{empty}

\begin{minipage}{0.5\linewidth}
	\large\textbf{CURSO DE ESCOAMENTOS\\MULTIFÁSICOS}
\end{minipage}
\begin{minipage}{0.5\linewidth}
	\flushright
	\includegraphics[height=18mm]{figs/logouerj.jpg}
	\hspace{.5cm}
	\includegraphics[height=16mm]{figs/logo_ppg-em.jpg}
\end{minipage}

\hrulefill

%\begin{minipage}[b]{0.5\linewidth}
\Large \color{NavyBlue} \textbf{ESCOAMENTO EM CANAL - Modelos 1D}\\
\color{Black}\\ % Title
%\vspace{-.5cm}
\normalsize \texttt{Preparado por: Leon Lima e Gustavo Anjos}\\
\normalsize \texttt{\today}
%\end{minipage}

%\begin{minipage}[b]{0.5\linewidth}
%\flushright
%\includegraphics[height=2cm]{../../imagens/logoeletronuclear.jpg}
%\hspace{.5cm}
%\includegraphics[height=2cm]{../../imagens/logouerj.jpg}\\
%\end{minipage}

\hrulefill
\begin{center}
	\begin{minipage}{0.8\linewidth}
		\footnotesize{\textbf{Resumo.} Este texto apresenta os pontos
		principais da modelagem matemática de escoamento incompressível,
		desenvolvido, em regimes laminar e turbulento, permanente e
		transiente, em canal aberto. Conceitos básicos de turbulência
		são sucintamente apresentados. Além disso, a discretização das
		equações de escoamento em canal 1D utilizando o método das
		diferenças finitas e o método de elementos finitos é apresenta
		para os perfis laminar e turbulento.}
	\end{minipage}
\end{center}

%\begin{multicols}{2}
{\footnotesize\tableofcontents}
\hrulefill\\
\onehalfspacing
\vspace{10mm}


\section{PERFIL LAMINAR}

\subsection{Regime laminar permanente}

Considere as componentes $x$ e $y$ das equações de Navier Stokes para escoamento incompressível (com viscosidade constante\footnote{Para um escoamento isotérmico, a hipótese de viscosidade constante é muito realista, uma vez que, embora ela dependa também da pressão do fluido (além da temperatura), as variações com a pressão são extremamente baixas.} e sem termo das forças de corpo):

\begin{subequations}
  \begin{eqnarray}
    \frac{\D u}{\D t}+u\frac{\D u}{\D x}+v\frac{\D u}{\D y}&=&-\frac{1}{\rho}\frac{\D p}{\D x}+\nu\left(\frac{\D^2u}{\D x^2}+\frac{\D^2u}{\D y^2}\right)\\
    \frac{\D v}{\D t}+u\frac{\D v}{\D x}+v\frac{\D v}{\D y}&=&-\frac{1}{\rho}\frac{\D p}{\D y}+\nu\left(\frac{\D^2v}{\D x^2}+\frac{\D^2v}{\D y^2}\right)
  \end{eqnarray}
  \label{eq:momentum}
\end{subequations}

\begin{equation}
  \frac{\D u}{\D x} + \frac{\D v}{\D y} = 0
  \label{eq:mass}
\end{equation}

Para o regime permanente do escoamento, $\frac{\D u}{\D t}=\frac{\D v}{\D t}=0$.

Suponha também que o escoamento seja desenvolvido, isto é:

\begin{equation}
  \frac{\D u}{\D x} = 0
\end{equation}
o que, devido a conservação de massa Eq.~(\ref{eq:mass}), implica em:

\begin{equation}
  \frac{\D v}{\D y} = 0
\end{equation}

Esse resultado significa que $v = \text{constante}$ ao longo de qualquer seção do canal. Mais do que isso: por causa das condições de contorno de não deslizamento, $v = 0$, em todo o domínio do escoamento.

Temos portanto:

\begin{subequations}
  \begin{eqnarray}
    -\frac{1}{\rho}\frac{\D p}{\D x}+\nu\frac{\D^2u}{\D y^2}&=&0\\
    -\frac{1}{\rho}\frac{\D p}{\D y}&=&0
  \end{eqnarray}
  \label{eq:sistema_laminar}
\end{subequations}

A condição de que a derivada parcial da pressão em relação a $y$ seja nula significa que $p = p(x)$, ou seja, o campo de pressão não depende da coordenada $y$. Este resultado significa que o escoamento incompressível, desenvolvido e estacionário em um canal cujos fluxos na direção $z$ são irrelevantes é modelado por

\begin{equation}
  \frac{\D p}{\D x}=\mu\frac{\D^2u}{\D y^2}
  \label{eq:perfil_laminar_edp}
\end{equation}

Uma vez que a pressão depende apenas de $x$ e a velocidade $u$ depende
apenas de $y$, podemos concluir que a pressão depende linearmente de
$x$, de maneira que $\D p/\D x$ é constante. Dado um gradiente de
pressão, a equação \ref{eq:perfil_laminar_edp} se torna uma EDO
homogênea de segunda ordem, cuja solução é o perfil laminar do
escoamento com as características citadas, expresso por:

\begin{equation}
  u(y) = \frac{1}{2\mu}\frac{\D p}{\D x}y^2+C_1y+C_2
\label{eq:sol_geral1}
\end{equation}

A Eq.(\ref{eq:sol_geral1}) fornece a solução geral do problema
representado pela Eq.~(\ref{eq:perfil_laminar_edp}). As condições de
contorno definem a natureza do problema. Tratamos aqui o caso particular
de não-escorregamento de partículas em contato com as paredes inferior
$y=0$ e superior $y=L$. Com isso, as condições de contorno para
velocidade se escrevem: 

\begin{eqnarray}
	u(y=0) &=& u(0) = 0\\
	u(y=L) &=& u(L) = 0
\label{eq:cc}
\end{eqnarray}

Os coeficientes $C_1$ e $C_2$ da Eq.~(\ref{eq:sol_geral1}) podem ser
encontrados através da imposição das condições de contorno, definindo
então a solução particular do problema:

\begin{eqnarray}
    C_1 &=& \frac{u(L)-u(0)}{L}-\frac{1}{2\mu}\frac{\D p}{\D x}L\\
    C_2 &=& u(0)
\end{eqnarray}

onde $L$ é a largura do canal. Se $u(0)=u(L)=0$, obtemos:

\begin{equation}
  u(y) = \frac{1}{2\mu}\frac{\D p}{\D x}\left(y^2-Ly\right)
  \label{eq:perfil_laminar}
\end{equation}

Para o problema do canal, a velocidade $U$ é máxima no centro $y=L/2$.
Considere $u(y=L/2)=U$. Para o perfil dado pela equação
\ref{eq:perfil_laminar}, temos portanto que:

\begin{equation}
  U = \frac{1}{2\mu}\frac{\D p}{\D x}\left(-\frac{L^2}{4}\right)
\end{equation}

o que resulta em:

\begin{equation}
  \frac{\D p}{\D x}=-8\frac{\mu U}{L^2}
\end{equation}

Definindo o número de Reynolds como $\Re=UL\rho/\mu$, obtemos

\begin{equation}
  \frac{\D p}{\D x}=-8\frac{\mu^2\Re}{\rho L^3}
  \label{eq:gradp}
\end{equation}

\subsection{Regime laminar transiente}

O regime transiente do mesmo escoamento pode ser representado por

\begin{equation}
    \frac{\D u}{\D t}=-\frac{1}{\rho}\frac{\D p}{\D x}+\nu\frac{\D^2u}{\D y^2}
    \label{eq:perfil_laminar_transiente}
\end{equation}

Note que agora a equação depende da densidade $\rho$ do fluido. De fato, para o regime transiente, a variação do perfil de velocidade com o tempo deve ser influenciada pela inércia do fluido, convergindo, no entanto, para um valor comum para qualquer valor de $\rho$. Note ainda que a pressão agora é também função do tempo, isto é, $p = p(x,t)$ e seu gradiente não pode mais ser calculado pela expressão \ref{eq:gradp}.



\section{PERFIL TURBULENTO}

Para números de Reynolds acima de um certo limite\footnote{Reynolds observou em experimento os regimes pelos quais um escoamento pode passar e quais os parâmetros influenciavam na transição. Suas conclusões em cima desse trabalho foram publicadas em 1883\cite{REYNOLDS83,DAVIDSON11}}, eventuais perturbações introduzidas ao escoamento podem gerar oscilações cujas amplitudes cresçam monotonicamente, tornando-o instável hidrodinamicamente e convertendo o regime laminar em turbulento. Escoamentos turbulentos são caracterizados por

\begin{itemize}
\item alto grau de mistura
\item riqueza de escalas
\item caos
\end{itemize}

Osborne Reynolds introduziu a abordagem estatística ao estudo de escoamentos turbulentos em 1895 \cite{REYNOLDS95}, segundo a qual o escoamento médio é resolvido. Para escoamentos quase estacionários, médias temporais podem ser usadas \cite{FREIRE02}. Matematicamente, o conceito introduzido por Reynolds consistia na média das equações de Navier-Stokes, cujo resultado são as equações RANS -- Reynolds Averaged Navier-Stokes. Para escoamentos incompressíveis, elas são dadas por

\begin{equation}
  \frac{\D\vec{v}}{\D t}+\vec{v}\cdot\grad\vec{v}=-\frac{1}{\rho}\grad p+\nu\nabla^2\vec{v}-\div\overline{\tilde{\vec{v}}\otimes\tilde{\vec{v}}}
  \label{eq:rans_incompressivel_1}
\end{equation}

Expandindo em coordenadas cartesianas temos

\begin{subequations}
  \begin{eqnarray}
    \frac{\D u}{\D t}+u\frac{\D u}{\D x}+v\frac{\D u}{\D y}+w\frac{\D u}{\D z}&=&-\frac{1}{\rho}\frac{\D p}{\D x}+\nu\left(\frac{\D^2u}{\D x^2}+\frac{\D^2u}{\D y^2}+\frac{\D^2u}{\D z^2}\right)-\left(\frac{\D}{\D x}\overline{\tilde{u}\tilde{u}}+\frac{\D}{\D y}\overline{\tilde{u}\tilde{v}}+\frac{\D}{\D z}\overline{\tilde{u}\tilde{w}}\right)\\
    \frac{\D v}{\D t}+u\frac{\D v}{\D x}+v\frac{\D v}{\D y}+w\frac{\D v}{\D z}&=&-\frac{1}{\rho}\frac{\D p}{\D y}+\nu\left(\frac{\D^2v}{\D x^2}+\frac{\D^2v}{\D y^2}+\frac{\D^2v}{\D z^2}\right)-\left(\frac{\D}{\D x}\overline{\tilde{v}\tilde{u}}+\frac{\D}{\D y}\overline{\tilde{v}\tilde{v}}+\frac{\D}{\D z}\overline{\tilde{v}\tilde{w}}\right)\\
    \frac{\D w}{\D t}+u\frac{\D w}{\D x}+v\frac{\D w}{\D y}+w\frac{\D w}{\D z}&=&-\frac{1}{\rho}\frac{\D p}{\D z}+\nu\left(\frac{\D^2w}{\D x^2}+\frac{\D^2w}{\D y^2}+\frac{\D^2w}{\D z^2}\right)-\left(\frac{\D}{\D x}\overline{\tilde{w}\tilde{u}}+\frac{\D}{\D y}\overline{\tilde{w}\tilde{v}}+\frac{\D}{\D z}\overline{\tilde{w}\tilde{w}}\right)
  \end{eqnarray}
\end{subequations}
É importante notar que o campo de velocidade $\vec{v}$ e suas componentes, bem como o campo de pressão $p$, expressam aqui os respectivos valores médios, enquanto que o campo $\tilde{\vec{v}}$ e suas componentes representam as flutuações em torno dos valores médios, conforme a decomposição de Reynolds.

A dissipação introduzida pelos termos não-lineares de flutuação pode ser interpretada como um campo de tensão adicional atuando no escoamento, representado pelo tensor de tensões turbulentas dado por $\tau_t=-\rho\overline{\tilde{\vec{v}}\otimes\tilde{\vec{v}}}$, cujo divergente é

\begin{equation}
  \div\tau_t = -\rho\div\overline{\tilde{\vec{v}}\otimes\tilde{\vec{v}}}
\end{equation}

A equação \ref{eq:rans_incompressivel_1} pode portanto ser reescrita como

\begin{equation}
  \frac{\D\vec{v}}{\D t}+\vec{v}\cdot\grad\vec{v}=-\frac{1}{\rho}\grad p+\frac{1}{\rho}\div\left(\mu\grad\vec{v}+\tau_t\right)
  \label{eq:rans_incompressivel_2}
\end{equation}

Esta formulação introduz novas variáveis ao modelo, indeterminando o sistema de equações. A estratégia clássica de fechamento do sistema é a aplicação da hipótese de Boussinesq, proposta em 1877\cite{FREIRE02}\footnote{O artigo de Reynolds propondo a decomposição dos campos em parcelas média e flutuante e dando origem ao hoje chamado tensor de Reynolds foi publicado somente em 1895. O que havia como base de conhecimento para Boussinesq fazer essa proposta em 1877?}, segundo a qual os processos de difusão da quantidade de movimento molecular e turbulento são análogos. Matematicamente, isso equivale a

\begin{equation}
  \tau_t = \mu_t\grad\vec{v}
\end{equation}
onde $\mu_t$ é dita viscosidade turbulenta, ou viscosidade de Boussinesq. É mais conveniente agora nos referirmos à viscosidade molecular $\mu_m$. No caso da viscosidade cinética temos portanto $\nu_t$ e $\nu_m$. Para a maioria dos números de Reynolds, $\nu_t$ é algumas ordens de grandeza superior a $\nu_m$, ou seja, as dissipações turbulentas são muito maiores do que as dissipações viscosas. Introduzindo o conceito de viscosidade turbulenta na equação \ref{eq:rans_incompressivel_2}, obtemos

\begin{equation}
  \frac{\D\vec{v}}{\D t}+\vec{v}\cdot\grad\vec{v}=-\frac{1}{\rho}\grad p+\div\left[(\nu_m+\nu_t)\grad\vec{v}\right]
  \label{eq:rans_incompressivel_boussinesq1}
\end{equation}

A viscosidade adicional $\mu_t$ pode ser interpretada como um acréscimo dos efeitos de dissipação ao escoamento, e deve estar associada às características do escoamento.

Para o escoamento desenvolvido no canal, a equação \ref{eq:rans_incompressivel_boussinesq1} se reduz a

\begin{equation}
  \frac{\D u}{\D t}=-\frac{1}{\rho}\frac{\D p}{\D x}+\frac{\D}{\D y}\left[(\nu_m+\nu_t)\frac{\D u}{\D y}\right]
  \label{eq:rans_incompressivel_boussinesq2}
\end{equation}

\subsection{Camada limite turbulenta}

A região do escoamento próxima à parede passa por transições importantes até chegar ao escoamento principal\footnote{A expressão em inglês para o escoamento principal seria ``bulk flow''.}. Essa região é a camada limite do escoamento. É bem aceito que a camada limite possui duas regiões distintas: uma adjacente à parede, na qual os efeitos viscosos predominam -- subcamada viscosa -- e uma seguinte na qual os efeitos turbulentos são mais importantes -- subcamada turbulenta. Cada uma delas possui um perfil de velocidade diferente. Essa composição é conhecida como a estrutura assintótica da camada limite turbulenta.

Na subcamada viscosa, a condição do escoamento pode ser descrita por

\begin{equation}
  \mu\frac{\D^2u}{\D y^2} = 0
\end{equation}

Após integração, chegamos a

\begin{equation}
  u = \frac{Cy}{\mu}
\end{equation}
Tendo em vista que a tensão cisalhante $\tau_w$ na parede é

\begin{equation}
  \tau_w = \mu\frac{\D u}{\D y}
\end{equation}
temos que

\begin{equation}
  u = \frac{\tau_wy}{\mu}
  \label{eq:subcamada_viscosa_dim}
\end{equation}
Finalmente, introduzindo uma velocidade $u_{\tau}$, denominada velocidade de atrito, definida por

\begin{equation}
  u_{\tau} = \sqrt{\frac{\tau_w}{\rho}}
\end{equation}
podemos adimensionalizar a equação \ref{eq:subcamada_viscosa_dim} definindo

\begin{subequations}
  \begin{eqnarray}
    u^+ &=& \frac{u}{u_{\tau}}\\
    y^+ &=& \frac{yu_{\tau}}{\nu}
  \end{eqnarray}  
\end{subequations}
De tal forma que, para a subcamada viscosa, temos

\begin{equation}
  u^+=y^+
\end{equation}

Para a subcamada turbulenta, é necessária uma avaliação das ordens de grandeza dos termos importantes, utilizando o conceito de comprimento de mistura. \citet{FREIRE02} descrevem o desenvolvimento da expressão para a região turbulenta da camada limite. Aqui, vamos nos limitar a dizer que o perfil de velocidade na subcamada turbulenta é expresso por

\begin{equation}
  u^+=\frac{1}{\kappa}\ln y^+ + B
\end{equation}
onde $\kappa$ é a constante de Von Kármán, sendo normalmente $\kappa=0,41$, e $B = 5$ é um valor constante bem aceito para escoamentos em parede, baseado em resultados experimentais. A figura 3.13 de \citet{FREIRE02} apresenta um conjunto de perfis para a camada limite turbulenta obtidos de experimentos diversos.

\subsection{Comprimento de mistura de Prandtl}

A viscosidade turbulenta $\nu_t$ pode ser determinada através de modelo algébrico ou de modelo a uma equação diferencial ou de modelo a duas equações diferenciais (neste se enquadram os modelos $\kappa$-$\epsilon$ e $\kappa$-$\omega$).

O modelo algébrico é baseado no conceito de comprimento de mistura, concebido por Ludwig Prandtl (ver \citet{FREIRE02}, capítulo 3), segundo o qual

\begin{equation}
  \nu_t = l_c^2\left|\frac{\D u}{\D y}\right|
\end{equation}
onde $l_c$ é o comprimento de mistura. Para escoamentos próximos a paredes sólidas (caso do escoamento no canal), 

\begin{subequations}
  \begin{eqnarray}
    l_c&=&\kappa y\text{, para }y\leq\delta\\
    l_c&=&\kappa \delta\text{, para }y>\delta
  \end{eqnarray}
\end{subequations}
onde $\kappa$ é a constante de Von Kármán, e $\delta$ é a espessura da camada limite.

Com essa definição, a variação do comprimento de mistura passa por uma descontinuidade da parede para o interior da camada limite. O comprimento de mistura pode ser calculado ainda com a aplicação de uma função de amortecimento. Normalmente é usada a função de amortecimento de Van Driest. Neste caso,

\begin{equation}
  l_c=D\kappa y\text{, para }y\leq\delta
\end{equation}
com

\begin{equation}
  D = 1-\exp\left(-y\frac{u_{\tau}}{A\nu}\right)
\end{equation}
onde $A = 26$.

Cabe ressaltar que a modelagem do escoamento através da introdução da viscosidade turbulenta é uma das formas de solução do Problema de Fechamento dos modelos RANS e é dito modelo de turbulência de primeira ordem. Existem modelos nos quais as componentes do tensor de Reynolds são decompostas, dando a origem a termos com produtos de três componentes de velocidade, classificados como modelos de turbulência de segunda ordem. Um outro ponto é que, no caso particular do escoamento no canal, considerando que ele possua largura $L$, o comprimento de mistura é dado por

\begin{subequations}
  \begin{eqnarray}
    l_c &=& \kappa y\text{, para }y\leq\delta\\
    l_c &=& \kappa \delta\text{, para }\delta<y<L-\delta\\
    l_c &=& \kappa(L-y)\text{, para }y\geq(L-\delta)
  \end{eqnarray}
\end{subequations}

\subsection{Discretização}

A forma discreta da equação \ref{eq:rans_incompressivel_boussinesq2} é igual à equação \ref{eq:perfil_laminar_discr}, isto é,

\begin{eqnarray}
  \frac{u_i^{n+1}-u_i^n}{\Delta t} = -\frac{1}{\rho}\frac{\D p}{\D x} &+& \alpha\frac{\nu_{i+1/2}(u_{i+1}^{n+1}-u_{i}^{n+1})-\nu_{i-1/2}(u_{i}^{n+1}-u_{i-1}^{n+1})}{\Delta y^2}+\nonumber\\ &+& (1-\alpha)\frac{\nu_{i+1/2}(u_{i+1}^n-u_{i}^n)-\nu_{i-1/2}(u_{i}^n-u_{i-1}^n)}{\Delta y^2}
  \label{eq:perfil_turbulento_discr}
\end{eqnarray}
Porém com $\nu = \nu_m+\nu_t$.


\typeout{ ====================================================================}
\typeout{ this is file mdf.tex, created at 10-Apr-2015               }
\typeout{ maintained by Gustavo Rabello dos Anjos                             }
\typeout{ e-mail: gustavo.rabello@gmail.com                                   }
\typeout{ ====================================================================}

\section{MÉTODO DE DIFERENÇAS FINITAS}

Laminar transiente:

\begin{equation}
	\frac{\partial u}{\partial t}
	=
	- \frac{1}{\rho} \frac{\partial p}{\partial x}
	+ \frac{\partial}{\partial y} \nu \frac{\partial u}{\partial y}
\end{equation}

A discretização em diferenças finitas unidimensional é feita para um
ponto genérico de malha $i$, para passo de tempo $\Delta T$ e distância
entre nós uniforme igual a $\Delta y$, a discretização da equação
\ref{eq:perfil_laminar_transiente} em diferenças finitas centradas é
expressa por:

\begin{eqnarray}
  \frac{u_i^{n+1}-u_i^n}{\Delta t} = -\frac{1}{\rho}\frac{\D p}{\D x} &+&
  \alpha\frac{\nu_{i+1/2}(u_{i+1}^{n+1}-u_{i}^{n+1})
  -\nu_{i-1/2}(u_{i}^{n+1}-u_{i-1}^{n+1})}{\Delta y^2} +\nonumber\\ &+&
  (1-\alpha)\frac{\nu_{i+1/2}(u_{i+1}^n-u_{i}^n)
  -\nu_{i-1/2}(u_{i}^n-u_{i-1}^n)}{\Delta y^2}
  \label{eq:perfil_turbulento_discr}
\end{eqnarray}

Note que $\alpha$ representa uma variável limitada por $0\leq\alpha\leq
1$ que ajusta o esquema discreto no tempo em explícito $\alpha=0$,
implícito $\alpha=1$ e de segunda ordem do tipo Crank-Nicholson para
$\alpha=1/2$. A viscosidade cinemática $\nu = \frac{\mu}{\rho}$ é
discretizada no ponto médio dos segmentos da malha em diferenças finitas
$i+1/2$ e $i-1/2$, tomando-se então $\nu_{i+1/2} =
(\nu_{i}+\nu_{i+1})/2$ e $\nu_{i-1/2} = (\nu_{i-1}+\nu_{i})/2$.

Para a montagem (``\emph{assembling}'') da matriz do termo difusivo A,
propoê-se um esquema dentro de um laço (``\emph{loop}'') do tipo:

\begin{itemize}
	\item{coluna j-1:}
		\begin{equation}
			-\alpha \frac{\nu_{i-1/2}}{\Delta y}
		\end{equation}
	\item{coluna j:}
		\begin{equation}
			\frac{1}{\Delta t} 
			+ \alpha \frac{\nu_{i+1/2}}{\Delta y}
			+ \alpha \frac{\nu_{i-1/2}}{\Delta y}
		\end{equation}
	\item{coluna j+1:}
		\begin{equation}
			-\alpha \frac{\nu_{i-1/2}}{\Delta y}
		\end{equation}
\end{itemize}

Para a montagem do vetor do lado direto:

\begin{itemize}
	\item{linha i:}
		\begin{eqnarray}
			\frac{u_i^{n}}{\Delta t} 
			-\frac{1}{\rho}\frac{\partial p}{\partial x} 
			(1-\alpha)\frac{\nu_{i+1/2}(u_{i+1}^n-u_{i}^n)
			-\nu_{i-1/2}(u_{i}^n-u_{i-1}^n)}{\Delta y^2}
			\label{eq:perfil_turbulento_discr}
		\end{eqnarray}
\end{itemize}

\subsection{Condições de Contorno}


\typeout{ ****************** End of file mdf.tex ****************** }


\typeout{ ====================================================================}
\typeout{ this is file mef.tex, created at 22-Feb-2015               }
\typeout{ maintained by Gustavo Rabello dos Anjos                             }
\typeout{ e-mail: gustavo.rabello@gmail.com                                   }
\typeout{ ====================================================================}


\section{MÉTODO DE ELEMENTOS FINITOS}









\typeout{ ****************** End of file mef.tex ****************** }


\typeout{ ====================================================================}
\typeout{ this is file particulas.tex, created at 10-Apr-2015               }
\typeout{ maintained by Gustavo Rabello dos Anjos                             }
\typeout{ e-mail: gustavo.rabello@gmail.com                                   }
\typeout{ ====================================================================}

\section{PARTÍCULAS}

Em dinâmica dos fluidos, a equação de Basset-Boussinesq-Oseen (BBO)
descreve o movimento e as forças que atuam em partículas pequenas para
escoamentos transiente com baixo número de Reynolds. Sua formulação pode
ser encontrada de diversas formas. 

\begin{equation}
	\frac{\pi}{6} \rho_p d_p^3 \frac{du_p}{dt} 
	=
	3 \pi \mu d_p (u_f-u_p)
	-
	\frac{\pi}{6} d_p^3 \nabla p
	+
	\frac{\pi}{12} \rho_f d_p^3 \frac{d}{dt}(u_f-u_p)
	+
	\frac{3}{2} d_p^2 \sqrt(\pi \rho_f \mu) \cdots
	+
	\sum F_k
\end{equation}

\noindent onde o primeiro termo do lado direto da equação representa o
arrasto, o segundo termo representa o gradiente de pressão, o terceiro
a força virtual, o quarto a força de Basset e o quinto termo representa
qualquer outra força que deva ser levada em consideração, como por
exemplo a força de gravidade.

\begin{equation}
	\sum F = ma = m \frac{D \mathbf{u}}{D t}
\end{equation}

Em $\sum F$ diversas forças podem ser consideradas no modelo dependendo
do grau de complexidade e precisão do esquema proposto. Neste seção,
consideraremos àquelas pertinentes ao desenvolvimento do um modelo
simplificado, cabendo ressaltar que a consideração de outras forças
varia de acordo com a aplicação em problemas mais complexos. Para
maiores informações, ao leitor é sugerido as seguintes referências:
\cite{crowe2012}, 

\begin{equation}
	\sum F = F_{lift} + F_{drag} + F_{gravity} + F_{virtual}
\end{equation}

\noindent onde $F_{gravity}$ representa a força gravitacional definida
como: 

\begin{equation}
	dF_{gravity}
	=
	\rho_p g dv
\end{equation}

ao integramos no volume ocupado por uma partícula qualquer, a força
gravitacional toma a forma de:

\begin{equation}
	dF_{gravity}
	=
	\rho_p \pi \frac{d_p^3}{6} g
\end{equation}

\noindent onde $dv=\pi d_p^3/6$. $F_{lift}$ representa a força de
sustentação (``lift'')











\typeout{ ****************** End of file particulas.tex ****************** }



\section{TESTES}

Uma implementação dos modelos descritos possibilitará os seguintes testes:

\begin{itemize}
   \item Para um mesmo problema, integre a solução final ao longo do
   domínio espacial para duas resoluções de malha diferentes. Tente
   observar o efeito da dissipação numérica. Quanto menor o número de
   pontos, maior é a quantidade de informação perdida na aproximação das
   derivadas. Um escoamento mais dissipativo (ainda que falsamente)
   resultará em valores menores de $\int_0^Ludy$.
   \item Os perfis turbulentos são mais ``achatados'' do que os
   laminares, o que é possível de ser explicado a partir da definição do
   comprimento de mistura. Uma vez que a intensidade da turbulência é
   proporcional ao número de Reynolds, é possível notar que o perfil se
   torna mais ``achatado'' para maiores números de Reynolds.
   \item A solução do perfil turbulento não representa o escoamento
   ``real''. De fato, uma vez que a solução das equações RANS fornece os
   campos médios, as flutuações do escoamento não são vistas pelos
   perfis obtidos da solução de \ref{eq:perfil_turbulento_discr}.
   \item Tente reproduzir os perfis da camada limite turbulenta. Calcule
   a solução permanente do escoamento turbulento e tente observar as
   subcamadas viscosa e turbulenta.
\end{itemize}



\singlespacing

\bibliographystyle{plainnat}
\bibliography{ref}

%\end{multicols}
\end{document}
