
\section{TESTES}

Uma implementação dos modelos descritos possibilitará os seguintes testes:

\begin{itemize}
   \item Para um mesmo problema, integre a solução final ao longo do domínio espacial para duas resoluções de malha diferentes. Tente observar o efeito da dissipação numérica. Quanto menor o número de pontos, maior é a quantidade de informação perdida na aproximação das derivadas. Um escoamento mais dissipativo (ainda que falsamente) resultará em valores menores de $\int_0^Ludy$.
   \item Os perfis turbulentos são mais ``achatados'' do que os laminares, o que é possível de ser explicado a partir da definição do comprimento de mistura. Uma vez que a intensidade da turbulência é proporcional ao número de Reynolds, é possível notar que o perfil se torna mais ``achatado'' para maiores números de Reynolds.
   \item A solução do perfil turbulento não representa o escoamento ``real''. De fato, uma vez que a solução das equações RANS fornece os campos médios, as flutuações do escoamento não são vistas pelos perfis obtidos da solução de \ref{eq:perfil_turbulento_discr}.
   \item Tente reproduzir os perfis da camada limite turbulenta. Calcule a solução permanente do escoamento turbulento e tente observar as subcamadas viscosa e turbulenta.
\end{itemize}

