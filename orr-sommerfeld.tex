\documentclass[a4paper,portuguese,10pt]{article}

\usepackage{setspace}
%\usepackage[hang,footnotesize]{caption}
\usepackage[utf8]{inputenc}
\usepackage{graphicx}
\usepackage{wrapfig}
%\usepackage{hyperref}
\usepackage[colorlinks=true]{hyperref}
\hypersetup{
  linkcolor=blue,
  citecolor=blue,
}
%\usepackage{setspace}
%\usepackage{abntcite}
\usepackage{amsfonts}
\usepackage{amsmath}
\usepackage{nonfloat}
\usepackage{semtrans}
\usepackage[margin=2cm]{geometry}
\usepackage[portuguese]{babel}
\usepackage[fixlanguage]{babelbib}
\selectbiblanguage{portuguese}
\usepackage[svgnames]{xcolor} % Specify colors by their 'svgnames'; list of all colors available: http://www.latextemplates.com/svgnames-colors
\usepackage{titlesec}
\usepackage[numbers]{natbib}
\usepackage{nomencl}
\usepackage{ifthen}
\usepackage{multicol} % use of multiple columns
\usepackage{fancyhdr}
\usepackage{draftwatermark}
\SetWatermarkText{PRELIMINAR}
\SetWatermarkScale{3}
%\SetWatermarkColor[rgb]{0,1,0}
\SetWatermarkLightness{.9}

\columnsep=8mm
%\columnseprule=1pt

\makenomenclature
% no build the list of symbols, use the following command:
% makeindex report_Leon_TUD_02-June.nlo -s nomencl.ist -o report_Leon_TUD_02-June.nls

\onehalfspacing


\newcommand{\Y}{\hat\Psi}
\newcommand{\Yb}{\overline\Psi}
\newcommand{\p}{\parallel}
\newcommand{\m}{\mid}
\newcommand{\del}{\bigtriangleup}
\newcommand{\lb}{\linebreak}
\newcommand{\nl}{\newline}
%\renewcommand{\div}{{\,\rm div}\,}
\renewcommand{\div}{\nabla\cdot}
%\newcommand{\grad}{\,\rm {grad}\,}
\newcommand{\grad}{\nabla}
\renewcommand{\S}{\sum_{i=1}^n}
\renewcommand{\D}{\partial}
\newcommand{\vecv}{\mathbf{v}}
\newcommand{\um}{\overline{u}}
\newcommand{\vm}{\overline{v}}
\newcommand{\wm}{\overline{w}}
\renewcommand{\pm}{\overline{p}}
\newcommand{\vecvm}{\overline{\mathbf{v}}}
\newcommand{\uf}{\tilde{u}}
\newcommand{\vf}{\tilde{v}}
\newcommand{\wf}{\tilde{w}}
\newcommand{\pf}{\tilde{p}}
\newcommand{\vecvf}{\tilde{\mathbf{v}}}
\newcommand{\RR}{\mathbb{R}}
\renewcommand{\vec}{\mathbf}
\newcommand{\esp}{\text{\hspace{2mm}}}
\newcommand{\dgs}{\textordmasculine}
\renewcommand{\max}{\operatorname{max}}
\newcommand{\Ra}{\operatorname{Ra}}
\renewcommand{\Re}{\operatorname{Re}}
\newcommand{\Gr}{\operatorname{Gr}}
\newcommand{\St}{\operatorname{St}}
\renewcommand{\Pr}{\operatorname{Pr}}
\newcommand{\CFL}{\operatorname{CFL}}
\newcommand{\tr}{\operatorname{tr}}
\newcommand{\e}{\operatorname{e}}

%\setlength{\hoffset}{-1cm}
%\setlength{\textwidth}{17cm}
%\setlength{\parskip}{1cm plus4mm minus3mm}
\setlength{\parskip}{2mm}
\setlength{\parindent}{0mm} % Default is 15pt.
\titleformat*{\section}{\normalsize\bfseries}
\titleformat*{\subsection}{\normalsize}

\renewcommand{\refname}{\normalsize{References}}
\renewcommand{\nomname}{List of Symbols}

\renewcommand{\nomgroup}[1]{%
\ifthenelse{\equal{#1}{R}}{\item[\textbf{Roman letters}]}{%
\ifthenelse{\equal{#1}{G}}{\item[Greek letters]}{}}}

\pagestyle{fancy}
\lhead{}
\rhead{\small{Equação de Orr-Sommerfeld}}


%%%%%%%%%%%%%%%%%%%%%%%%%%%%%%%%%%%%%%%%%%%%%%%%%%%%%%%%%%%%%%%%%%%%%%%%%%%%%%%%%%
\begin{document}

\thispagestyle{empty}

\begin{minipage}{0.5\linewidth}
\flushleft
\includegraphics[height=20mm]{./figs/logo_uerj_PB.png}
\end{minipage}
\begin{minipage}{0.5\linewidth}
\flushright
\includegraphics[height=18mm]{./figs/logo_ppg-em.jpg}
\end{minipage}

\hrulefill


%\begin{center}
%\includegraphics[scale=.3]{../../imagens/logouerj.jpg}\\
%\vspace{5mm}
%{\Large\texttt{Faculdade de Engenharia Mecânica - UERJ}}\\
%\vspace{10mm}
%\end{center}


%\vspace{5mm}
\Large \color{NavyBlue} \textbf{EQUAÇÃO DE ORR-SOMMERFELD}\\
\color{Black} % Title
\normalsize \texttt{Preparado por: Leon Lima}\\%[0.5cm] % Author(s)
\normalsize \texttt{\today}
\vspace{-2mm}

\setcounter{tocdepth}{1}
\hrulefill\\

%\vspace*{-5mm}
%\tableofcontents
%\hrulefill

\section{Introdução}

Escoamentos turbulentos são gerados por perturbações introduzidas no regime laminar. Em escoamentos laminares instáveis, eventuais perturbações geram oscilações crescentes que conduzem à turbulência. A estabilidade hidrodinâmica do regime transicional entre o escoamento laminar e o turbulento é, portanto, de grande relevância, tendo sido alvo de muitos estudos desde o início do século XX \cite{SCHLICHTING00}.

A equação de Orr-Sommerfeld modela o comportamento de um escoamento laminar paralelo incompressível quanto às oscilações geradas a partir de uma perturbação.

O texto que segue busca apresentar passo a passo o procedimento matemático para se chegar à equação de Orr-Sommerfeld, abrindo mão do rigor matemático. Parte deste texto teve como base o livro de \citet{SCHLICHTING00}.

\section{Desenvolvimento}

Seja $\vecv = u\vec{i}+v\vec{j}$ o campo de velocidade de um escoamento incompressível, em duas dimensões, com propriedades constantes e sem o termo de forças. As equações de Navier-Stokes para este escoamento é expressa por

\begin{equation}
  \frac{\D u}{\D x}+\frac{\D v}{\D y} = 0
  \label{eq_cont}
\end{equation}

\begin{equation}
  \frac{\D u}{\D t}+u\frac{\D u}{\D x}+v\frac{\D u}{\D y}=-\frac{1}{\rho}\frac{\D p}{\D x}+\nu\left(\frac{\D^2u}{\D x^2}+\frac{\D^2u}{\D y^2}\right)
  \label{eq_qdm_u}
\end{equation}

\begin{equation}
  \frac{\D v}{\D t}+u\frac{\D v}{\D x}+v\frac{\D v}{\D y}=-\frac{1}{\rho}\frac{\D p}{\D y}+\nu\left(\frac{\D^2v}{\D x^2}+\frac{\D^2v}{\D y^2}\right)
  \label{eq_qdm_v}
\end{equation}

Seja $\Psi$ a função de corrente deste escoamento, i.e.,

\begin{subequations}
  \begin{eqnarray}
    u &=& \frac{\D\Psi}{\D y}\\
    v &=& -\frac{\D\Psi}{\D x}
  \end{eqnarray}
\end{subequations}

Na sequência do desenvolvimento, as derivadas parciais de um dado campo serão denotadas através de um subescrito indicando a variável em relação à qual o campo foi derivado, e.g., $\D\Psi/\D t$ será denotada por $\Psi_t$, e $\D/\D x(\D\Psi/\D t)$ será $\Psi_{tx}$.

As equações de Navier-Stokes (eqs. \ref{eq_cont}-\ref{eq_qdm_v}) podem ser escritas em termos da função de corrente, resultando em

\begin{subequations}
  \begin{eqnarray}
    \label{eq_psi_u_01}
    \Psi_{yt}+\Psi_y\Psi_{yx}-\Psi_x\Psi_{yy}&=&-\frac{1}{\rho}p_x+\nu\left[\Psi_{yxx}+\Psi_{yyy}\right]\\
    \label{eq_psi_v_01}
    -\Psi_{xt}-\Psi_y\Psi_{xx}+\Psi_x\Psi_{xy}&=&-\frac{1}{\rho}p_y+\nu\left[-\Psi_{xxx}-\Psi_{xyy}\right]
  \end{eqnarray}
\end{subequations}

Derivando a equação \ref{eq_psi_u_01} em relação a $y$ e a equação \ref{eq_psi_v_01} em relação a $x$, obtemos

\begin{subequations}
  \begin{eqnarray}
    \label{eq_psi_u_02}
    \Psi_{yty}+\Psi_{yy}\Psi_{yx}+\Psi_{y}\Psi_{yxy}-\Psi_{xy}\Psi_{yy}-\Psi_{yyy}\Psi_{x}&=&-\frac{1}{\rho}p_{xy}+\nu\left[\Psi_{yxxy}+\Psi_{yyyy}\right]\\
    \label{eq_psi_v_02}
    -\Psi_{xtx}-\Psi_{yx}\Psi_{xx}-\Psi_{y}\Psi_{xxx}+\Psi_{xx}\Psi_{xy}+\Psi_{xyx}\Psi_{x}&=&-\frac{1}{\rho}p_{yx}+\nu\left[-\Psi_{xxxx}-\Psi_{xyyx}\right]
  \end{eqnarray}
\end{subequations}

Subtraindo a equação \ref{eq_psi_v_02} da \ref{eq_psi_u_02}, chegamos a

\begin{equation}
  \nabla^2\Psi_t+\Psi_{yx}\nabla^2\Psi+\Psi_y\nabla^2\Psi_x-\Psi_{xy}\nabla^2\Psi-\Psi_x\nabla^2\Psi_y=\nu\left[\nabla^2\Psi_{xx}+\nabla^2\Psi_{yy}\right]
  \label{eq_psi_01}
\end{equation}
onde $\nabla^2$ representa o operador laplaciano. Considerando que $\Psi_{yx}\nabla^2\Psi=\Psi_{xy}\nabla^2\Psi$, a equação \ref{eq_psi_01} pode ser reescrita como

\begin{equation}
  \nabla^2\Psi_t+\Psi_y\nabla^2\Psi_x-\Psi_x\nabla^2\Psi_y=\nu\nabla^2\left(\nabla^2\Psi\right)
  \label{eq_psi_02}
\end{equation}

Considere agora a introdução de uma perturbação no escoamento através da função de corrente, a qual pode ser decomposta num campo base $\Yb$ e uma flutuação $\Y$, i.e.,

\begin{equation}
  \Psi = \Yb+\Y
\end{equation}

Substituindo na equação \ref{eq_psi_02}, obtemos

\begin{equation}
  \nabla^2\Yb_t+\nabla^2\Y_t+(\Yb_y+\Y_y)\nabla^2(\Yb_x+\Y_x)-(\Yb_x+\Y_x)\nabla^2(\Yb_y+\Y_y)=\nu\nabla^2\left(\nabla^2\Yb\right)+\nu\nabla^2\left(\nabla^2\Y\right)
  \label{eq_psi_03}
\end{equation}
Considere agora a hipótese de que o escoamento base, com campo de velocidade $\vec{V}=U\vec{i}+V\vec{j}$, seja desenvolvido, i.e., $U=U(y)$, resultando em $\D U/\D x=0$. Supondo condição de não-deslizamento na parede, pela conservação de massa, temos que $V=0$, o que implica em $\Yb_x=0$. Levando em conta que o escoamento base $\Yb$ é solução para a equação \ref{eq_psi_02}, a equação \ref{eq_psi_03} pode ser dada por

\begin{equation}
  \nabla^2\Y_t+\Yb_y\nabla^2\Y_x+\Y_y\nabla^2\Y_x-\Y_x\nabla^2\Yb_y-\Y_x\nabla^2\Y_y=\nu\nabla^2\left(\nabla^2\Y\right)
  \label{eq_psi_04}
\end{equation}

Uma outra hipótese é a de pequenas perturbações. Isso faz com que os dois termos da equação \ref{eq_psi_04} com produto entre flutuações possam ser desprezados, resultando em

\begin{equation}
  \nabla^2\Y_t+\Yb_y\nabla^2\Y_x-\Y_x\nabla^2\Yb_y=\nu\nabla^2\left(\nabla^2\Y\right)
  \label{eq_psi_05}
\end{equation}
o que, levando em conta que $\Yb_y=U$ e que $\nabla^2U = U''$ (uma vez que $\D U/\D x=0$), pode ser reescrito como

\begin{equation}
  \nabla^2\Y_t+U\nabla^2\Y_x-\Y_xU''=\nu\nabla^2\left(\nabla^2\Y\right)
  \label{eq_psi_06}
\end{equation}

Considere que a perturbação $\Y$ consista de uma onda simples (único modo) propagando na direção $x$, dada por

\begin{equation}
  \Y(x,y,t)=\phi(y)\e^{i(kx-\omega t)}
  \label{eq_perturbacao}
\end{equation}
onde $k$ é um número real tal que $2\pi/k$ é o comprimento de onda da perturbação, e $\omega$ é um número complexo cuja parte real $\omega_r$ é a frequência do modo e a parte imaginária $\omega_i$ é o fator de amplificação. Se $\omega_i>0$, o escoamento laminar é instável, enquanto que, se $\omega_i<0$, ele é estável.

Vamos agora escrever cada uma das parcelas da equação \ref{eq_psi_06} introduzindo a perturbação definida por \ref{eq_perturbacao}.

\begin{subequations}
  \begin{eqnarray}
    \nabla^2\Y_t&=&(k^2\phi-\phi'')i\omega\e^{i(kx-\omega t)}\\
    U\nabla^2\Y_x&=&U(\phi''-k^2\phi)ik\e^{i(kx-\omega t)}\\
    \Y_xU''&=&ikU''\phi\e^{i(kx-\omega t)}\\
    \nu\nabla^2(\nabla^2\Y)&=&(-2k^2\phi''+k^4\phi+\phi'''')\e^{i(kx-\omega t)}
  \end{eqnarray}
\end{subequations}
Como $\e^{i(kx-\omega t)}$ é comum a todos os termos, a equação \ref{eq_psi_06} resulta em

\begin{equation}
  (k^2\phi-\phi'')i\omega+U(\phi''-k^2\phi)ik-U''ik\phi=\nu(-2k^2\phi''+k^4\phi+\phi'''')
  \label{eq_phi_01}
\end{equation}
dividindo a equação \ref{eq_phi_01} por $ik$, obtemos

\begin{equation}
  (k\phi-\phi''/k)\omega+U(\phi''-k^2\phi)-U''\phi=\frac{\nu}{ik}(-2k^2\phi''+k^4\phi+\phi'''')
  \label{eq_phi_02}
\end{equation}
Definindo $c=w/k$ e rearranjando, chegamos a

\begin{equation}
  (U-c)(\phi''-k^2\phi)-U''\phi=\frac{\nu}{ik}(-2k^2\phi''+k^4\phi+\phi'''')
\end{equation}

Esta equação diferencial de perturbação é o ponto de partida para a teoria de estabilidade de escoamentos laminares, e é chamada de Equação de Orr-Sommerfeld, em honra a William McFadden Orr (1907) e Arnold Sommerfeld (1908). Em sua forma adimensional, ela é dada por

\begin{equation}
  (U-c)(\phi''-k^2\phi)-U''\phi=\frac{1}{ik\Re}(-2k^2\phi''+k^4\phi+\phi'''')
\end{equation}
Aqui, evidentemente, as variáveis devem ser interpretadas como adimensionalizadas.

\section{Problema de auto-valor}

É usual que se possa controlar o comprimento de onda da perturbação $2\pi/k$, e que o $\Re$ seja um parâmetro conhecido do escoamento, de maneira que a solução da equação de Orr-Sommerfeld forneça, para um par $(k,\Re)$, o valor de $\omega$, através do qual a estabilidade do escoamento laminar base é avaliada.

A equação de Orr-Sommerfeld pode ser rearranjada de tal forma que obtemos

\begin{equation}
  -\frac{1}{ik\Re}\phi''''+\left(U-c+\frac{2k}{i\Re}\right)\phi''+\left(ck^2-Uk^2-U+\frac{2k}{i\Re}\right)\phi=0
  \label{eq_o-s_02}
\end{equation}

Discretizando o domínio de $U$  e $\phi$, as derivadas $\phi''''$ e $\phi''$ podem ser aproximadas, em diferenças finitas centradas de 2ª ordem, respectivamente por

\begin{equation}
  \phi'''' \approx \frac{\phi_{i-2}-4\phi_{i-1}+6\phi_i-4\phi_{i+1}+\phi_{i+2})}{\Delta y^4}
\end{equation}

\begin{equation}
  \phi'' \approx \frac{\phi_{i-1}-2\phi_i+\phi_{i+1}}{\Delta y^2}
\end{equation}
onde $i$ é o índice identificador do nó. A forma discreta da equação \ref{eq_o-s_02} pode ser expressa na forma matricial da seguinte forma:

\begin{equation}
  -\frac{1}{ik\Re}\vec{M}\phi+\left(U-c+\frac{2k}{i\Re}\right)\vec{N}\phi+\left(ck^2-Uk^2-U+\frac{2k}{i\Re}\right)\phi=0
  \label{eq_o-s_discreta_01}
\end{equation}

Definindo a matriz $\vec{A}$ tal que 

\begin{equation}
\vec{A}=-\frac{1}{ik\Re}\vec{M}+\left(U-c+\frac{2k}{i\Re}\right)\vec{N}
\end{equation}
e o campo escalar $\lambda$ tal que

\begin{equation}
\lambda=-\left(ck^2-Uk^2-U+\frac{2k}{i\Re}\right)
\end{equation}
podemos estabelecer o problema de auto-valor expresso por

\begin{equation}
  \vec{A}\phi = \lambda\phi
\end{equation}
em que $\lambda$ são os auto-valores da matriz $\vec{A}$.

\singlespacing
%\printnomenclature

\bibliographystyle{plainnat}
\bibliography{ref}

\end{document}
