\typeout{ ====================================================================}
\typeout{ this is file mdf.tex, created at 10-Apr-2015               }
\typeout{ maintained by Gustavo Rabello dos Anjos                             }
\typeout{ e-mail: gustavo.rabello@gmail.com                                   }
\typeout{ ====================================================================}

\section{MÉTODO DE DIFERENÇAS FINITAS}

\subsection{Laminar permanente}

\begin{equation}
	\frac{\partial}{\partial y} \nu \frac{\partial u}{\partial y}
	=
	\frac{1}{\rho} \frac{\partial p}{\partial x}
\end{equation}

A discretização em diferenças finitas unidimensional é feita para um
ponto genérico de malha $i$ e distância entre nós uniforme igual a
$\Delta y$, a discretização da equação \ref{eq:perfil_laminar_edp} em
diferenças finitas centradas é expressa por:

\begin{eqnarray}
	\frac{\nu_{i+1/2}(u_{i+1}-u_{i})
	-
	\nu_{i-1/2}(u_{i}-u_{i-1})}{\Delta y^2} 
	=
	\frac{1}{\rho}\frac{\D p}{\D x} 
\label{eq:perfil_laminar_discr}
\end{eqnarray}

Note que a viscosidade cinemática $\nu = \mu \ \rho$ é discretizada no
ponto médio dos segmentos da malha em diferenças finitas $i+1/2$ e
$i-1/2$, tomando-se então $\nu_{i+1/2} = (\nu_{i}+\nu_{i+1})/2$ e
$\nu_{i-1/2} = (\nu_{i-1}+\nu_{i})/2$.

Para a montagem (``\emph{assembling}'') da matriz do termo difusivo A da
equação em regime permanente, separe-se os coeficientes $i-1$, $i$ e
$i+1$ e organiza-se em colunas, uma vez que esta matriz é simétrica,
dentro de um laço (``\emph{loop}'') do tipo:

\begin{itemize}
	\item{coluna j-1:}
		\begin{equation}
			\frac{\nu_{i-1/2}}{\Delta y^2}
		\end{equation}
	\item{coluna j:}
		\begin{equation}
			\frac{-\nu_{i+1/2}-\nu_{i-1/2}}{\Delta y^2}
		\end{equation}
	\item{coluna j+1:}
		\begin{equation}
			\frac{\nu_{i+1/2}}{\Delta y^2}
		\end{equation}
\end{itemize}

Para a montagem do vetor do lado direto:

\begin{itemize}
	\item{linha i:}
		\begin{eqnarray}
			\frac{1}{\rho}\frac{\partial p}{\partial x} 
		\end{eqnarray}
\end{itemize}


\subsection{Laminar transiente}

\begin{equation}
	\frac{\partial u}{\partial t}
	=
	- \frac{1}{\rho} \frac{\partial p}{\partial x}
	+ \frac{\partial}{\partial y} \nu \frac{\partial u}{\partial y}
\end{equation}

A discretização em diferenças finitas unidimensional é da mesma forma
que no caso de escoamento laminar estacionário, porém com a inclusão da
derivada transiente ($\partial /\partial t$) na equação. Com isso, para
um ponto genérico de malha $i$, para passo de tempo $\Delta T$ e
distância entre nós uniforme igual a $\Delta y$, a discretização da
equação \ref{eq:perfil_laminar_transiente} em diferenças finitas
centradas é expressa por:

\begin{eqnarray}
  \frac{u_i^{n+1}-u_i^n}{\Delta t} = -\frac{1}{\rho}\frac{\D p}{\D x} &+&
  \alpha\frac{\nu_{i+1/2}(u_{i+1}^{n+1}-u_{i}^{n+1})
  -\nu_{i-1/2}(u_{i}^{n+1}-u_{i-1}^{n+1})}{\Delta y^2} +\nonumber\\ &+&
  (1-\alpha)\frac{\nu_{i+1/2}(u_{i+1}^n-u_{i}^n)
  -\nu_{i-1/2}(u_{i}^n-u_{i-1}^n)}{\Delta y^2}
  \label{eq:perfil_turbulento_discr}
\end{eqnarray}

Note que $\alpha$ representa uma variável limitada por $0\leq\alpha\leq
1$ que ajusta o esquema discreto no tempo em explícito $\alpha=0$,
implícito $\alpha=1$ e de segunda ordem do tipo Crank-Nicholson para
$\alpha=1/2$. A viscosidade cinemática $\nu = \mu \ \rho$ é
discretizada no ponto médio dos segmentos da malha em diferenças finitas
$i+1/2$ e $i-1/2$, tomando-se então $\nu_{i+1/2} =
(\nu_{i}+\nu_{i+1})/2$ e $\nu_{i-1/2} = (\nu_{i-1}+\nu_{i})/2$.

Para a montagem (``\emph{assembling}'') da matriz do termo difusivo A da
equação em regime transiente, propoê-se um esquema dentro de um laço
(``\emph{loop}'') do tipo:

\begin{itemize}
	\item{coluna j-1:}
		\begin{equation}
			-\alpha \frac{\nu_{i-1/2}}{\Delta y}
		\end{equation}
	\item{coluna j:}
		\begin{equation}
			\frac{1}{\Delta t} 
			+ \alpha \frac{\nu_{i+1/2}}{\Delta y}
			+ \alpha \frac{\nu_{i-1/2}}{\Delta y}
		\end{equation}
	\item{coluna j+1:}
		\begin{equation}
			-\alpha \frac{\nu_{i-1/2}}{\Delta y}
		\end{equation}
\end{itemize}

Para a montagem do vetor do lado direto:

\begin{itemize}
	\item{linha i:}
		\begin{eqnarray}
			\frac{u_i^{n}}{\Delta t} 
			-\frac{1}{\rho}\frac{\partial p}{\partial x} 
			(1-\alpha)\frac{\nu_{i+1/2}(u_{i+1}^n-u_{i}^n)
			-\nu_{i-1/2}(u_{i}^n-u_{i-1}^n)}{\Delta y^2}
			\label{eq:perfil_turbulento_discr}
		\end{eqnarray}
\end{itemize}

\subsection{Condições de Contorno}

\subsection{Turbulento transiente}




\typeout{ ****************** End of file mdf.tex ****************** }

