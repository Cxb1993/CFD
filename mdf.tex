\typeout{ ====================================================================}
\typeout{ this is file mdf.tex, created at 10-Apr-2015               }
\typeout{ maintained by Gustavo Rabello dos Anjos                             }
\typeout{ e-mail: gustavo.rabello@gmail.com                                   }
\typeout{ ====================================================================}

\section{MÉTODO DE DIFERENÇAS FINITAS}

\subsection{Laminar permanente}

\begin{equation}
	\frac{\partial}{\partial y} \left(\nu \frac{\partial u}{\partial y}\right)
	=
	\frac{1}{\rho} \frac{\partial p}{\partial x}
	\label{eq:perfil_laminar_perm}
\end{equation}

A discretização em diferenças finitas unidimensional é feita para um
ponto genérico de malha $i$ e distância entre nós uniforme igual a
$\Delta y$, a discretização da equação \ref{eq:perfil_laminar_edp} em
diferenças finitas centradas é expressa por:

\begin{eqnarray}
	\frac{\nu_{i+1/2}(u_{i+1}-u_{i})
	-
	\nu_{i-1/2}(u_{i}-u_{i-1})}{\Delta y^2} 
	=
	\frac{1}{\rho}\frac{\D p}{\D x} 
\label{eq:perfil_laminar_discr}
\end{eqnarray}

Note que a viscosidade cinemática $\nu = \mu / \rho$ é discretizada no
ponto médio dos segmentos da malha em diferenças finitas $i+1/2$ e
$i-1/2$, tomando-se então $\nu_{i+1/2} = (\nu_{i}+\nu_{i+1})/2$ e
$\nu_{i-1/2} = (\nu_{i-1}+\nu_{i})/2$.

O sistema de equações resultante é linear, e pode ser representado matricialmente por

\begin{equation}
	\vec{A}\vec{u}=\vec{b}
	\label{eq:sistema_linear}
\end{equation}
onde, considerando que a malha possua $m$ pontos nodais,

\begin{equation}
	\vec{A} = \left[
	\begin{array}{cccc}
		A_{1,1} & A_{1,2} & \dots & A_{1,m}\\
		A_{2,1} & A_{2,2} & \dots & A_{2,m}\\
		\vdots  & \vdots  & \ddots  & \vdots\\
		A_{m,1} & A_{m,2} & \dots & A_{m,m}
	\end{array}
	\right]
\end{equation}

\begin{equation}
	\vec{u} = \left[
	\begin{array}{c}
		u_{1}\\
		u_{2}\\
		\vdots\\
		u_{m}
	\end{array}
	\right]
\end{equation}

\begin{equation}
	\vec{b} = \left[
	\begin{array}{c}
		b_{1}\\
		b_{2}\\
		\vdots\\
		b_{m}
		\end{array}
	\right]
\end{equation}

Para uma linha $i$ e coluna $j$ genéricas, a montagem (\emph{assembling}) da matriz do termo difusivo $\vec{A}$ da
equação em regime permanente é feita organizando-se os coeficientes de $u_{i-1}$, $u_i$ e
$u_{i+1}$ em colunas %, uma vez que esta matriz é simétrica,
dentro de um laço (\emph{loop}) do tipo

\begin{itemize}
	\item{coluna $j-1$:}
		\begin{equation}
			A_{i,j-1} = \frac{\nu_{i-1/2}}{\Delta y^2}
		\end{equation}
	\item{coluna $j$:}
		\begin{equation}
			A_{i,j} = \frac{-\nu_{i+1/2}-\nu_{i-1/2}}{\Delta y^2}
		\end{equation}
	\item{coluna $j+1$:}
		\begin{equation}
			A_{i,j+1} = \frac{\nu_{i+1/2}}{\Delta y^2}
		\end{equation}
\end{itemize}

Para a montagem do vetor do lado direto $\vec{b}$:

\begin{itemize}
	\item{linha $i$:}
		\begin{eqnarray}
			b_i=\frac{1}{\rho}\frac{\partial p}{\partial x} 
		\end{eqnarray}
\end{itemize}

A solução do sistema linear \ref{eq:sistema_linear} fornece o perfil laminar permanente de velocidade no canal.

\subsection{Laminar transiente}

\begin{equation}
	\frac{\partial u}{\partial t}
	=
	- \frac{1}{\rho} \frac{\partial p}{\partial x}
	+ \frac{\partial}{\partial y}\left( \nu \frac{\partial u}{\partial y}\right)
\end{equation}

A discretização em diferenças finitas unidimensional ocorre da mesma forma
que no caso de escoamento laminar estacionário, porém com a inclusão da
derivada transiente ($\partial /\partial t$) na equação. Com isso, para
um ponto genérico de malha $i$, para passo de tempo $\Delta t$ e
distância entre nós uniforme igual a $\Delta y$, a discretização da
equação \ref{eq:perfil_laminar_transiente} em diferenças finitas
centradas é expressa por:

\begin{eqnarray}
  \frac{u_i^{n+1}-u_i^n}{\Delta t} = -\frac{1}{\rho}\frac{\D p}{\D x} &+&
  \alpha\frac{\nu_{i+1/2}(u_{i+1}^{n+1}-u_{i}^{n+1})
  -\nu_{i-1/2}(u_{i}^{n+1}-u_{i-1}^{n+1})}{\Delta y^2} +\nonumber\\ &+&
  (1-\alpha)\frac{\nu_{i+1/2}(u_{i+1}^n-u_{i}^n)
  -\nu_{i-1/2}(u_{i}^n-u_{i-1}^n)}{\Delta y^2}
  \label{eq:perfil_turbulento_discr}
\end{eqnarray}
onde $\alpha$ representa uma variável limitada por $0\leq\alpha\leq
1$ que ajusta o esquema discreto no tempo em explícito $\alpha=0$,
implícito $\alpha=1$ e de segunda ordem do tipo Crank-Nicholson para
$\alpha=1/2$. A viscosidade cinemática $\nu = \mu / \rho$ é
discretizada no ponto médio dos segmentos da malha em diferenças finitas
$i+1/2$ e $i-1/2$, tomando-se então $\nu_{i+1/2} =
(\nu_{i}+\nu_{i+1})/2$ e $\nu_{i-1/2} = (\nu_{i-1}+\nu_{i})/2$.

Para o \emph{assembling} da matriz do termo difusivo $\vec{A}$ da
equação em regime transiente, propõe-se um esquema dentro de um \emph{loop} do tipo (para linha $i$):

\begin{itemize}
	\item{coluna $j-1$:}
		\begin{equation}
			A_{i,j-1} = -\alpha \frac{\nu_{i-1/2}}{\Delta y}
		\end{equation}
	\item{coluna $j$:}
		\begin{equation}
			A_{i,j} = \frac{1}{\Delta t} 
			+ \alpha \frac{\nu_{i+1/2}}{\Delta y}
			+ \alpha \frac{\nu_{i-1/2}}{\Delta y}
		\end{equation}
	\item{coluna $j+1$:}
		\begin{equation}
			A_{i,j+1} = -\alpha \frac{\nu_{i+1/2}}{\Delta y}
		\end{equation}
\end{itemize}

Para a montagem do vetor do lado direto:

\begin{itemize}
	\item{linha $i$:}
		\begin{eqnarray}
			b_i = \frac{u_i^{n}}{\Delta t} 
			-\frac{1}{\rho}\frac{\partial p}{\partial x} 
			+(1-\alpha)\frac{\nu_{i+1/2}(u_{i+1}^n-u_{i}^n)
			-\nu_{i-1/2}(u_{i}^n-u_{i-1}^n)}{\Delta y^2}
			\label{eq:perfil_turbulento_discr}
		\end{eqnarray}
\end{itemize}

A adoção do esquema completamente explícito ($\alpha=0$) impõe um limite quanto ao valor do incremento de tempo $\Delta t$, acima do qual a formulação se torna numericamente instável. Este limite pode ser determinado a partir do método de análise de Von Neumann. O resultado da análise considerando viscosidade constante exige que

\begin{equation}
	\dt\leq 2\left(\frac{1}{\rho}\frac{\D p}{\D x}+\frac{4\nu}{\dy^2}\right)^{-1}
\end{equation}

A solução do sistema linear resultante fornece o campo laminar de velocidade para o passo de tempo $n$.

%segundo o qual o erro propagado através da solução pode ser expandido em série de Fourier. A análise é feita para um modo genérico do erro. Efetuando a transformação $u_{i+k}^{n+m}=g^me^{ik\zeta}$ para a equação \ref{eq:perfil_laminar_discr}, considerando viscosidade constante, obtém-se
%
%\begin{equation}
%	\frac{g-1}{\Delta t}=-\frac{1}{\rho}\frac{\D p}{\D x}+\nu\frac{e^{ik}+e^{-ik}-2}{\Delta y^2}
%\end{equation}
%que pode ser reescrita como
%
%\begin{equation}
%	\frac{g-1}{\Delta t}=-\frac{1}{\rho}\frac{\D p}{\D x}+2\nu\frac{\cos\zeta-1}{\Delta y^2}
%\end{equation}
%
%A condição para estabilidade consiste em $|g|\leq 1$, consequentemente,
%
%\begin{equation}
%	|g|=\left|-\frac{\Delta t}{\rho}\frac{\D p}{\D x}+2\nu(\cos\zeta-1)\frac{\Delta t}{\Delta y^2}+1\right|\leq 1
%\end{equation}

\subsection{Condições de Contorno}

Para que o problema tenha solução única, condições de contorno devem ser impostas. No escoamento em canal, é razoável supor condição de não deslizamento nas paredes, i.e.,

\begin{subequations}
	\begin{eqnarray}
		u(y=0) &=& 0\\
		u(y=L) &=& 0
	\end{eqnarray}
\end{subequations}

No contexto do problema discreto, com malha deslocada, esta condição exige que

\begin{subequations}
	\begin{eqnarray}
		u_1 + u_2 &=& 0\\
		u_m + u_{m-1} &=& 0
	\end{eqnarray}
\end{subequations}

De tal forma que

\begin{subequations}
	\begin{eqnarray}
	A_{1,1} &=& 1\\
	A_{1,2} &=& 1\\
	A_{m,m} &=& 1\\
	A_{m,m-1} &=& 1\\
	b_1 &=& 0\\
	b_m &=& 0
	\end{eqnarray}
\end{subequations}


%\subsection{Turbulento transiente}


\typeout{ ****************** End of file mdf.tex ****************** }

