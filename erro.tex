\typeout{ ====================================================================}
\typeout{ this is file erro.tex, created at 18-Aug-2016                       }
\typeout{ maintained by Gustavo Rabello dos Anjos                             }
\typeout{ e-mail: gustavo.rabello@gmail.com                                   }
\typeout{ ====================================================================}


\section{COMO CALCULAR ERRO}

Muitas vezes desejamos calcular em que distância uma aproximação se
encontra do valor exato da quantidade que estamos medindo. Esta
aproximação muitas vezes está associada ao tipo de metodologia que
estamos empregando ou simplesmente pelo fato de aproximarmos o valor de
uma constante em uma equação. O significado de um erro cometido está
estritamente relacionado à magnitude da quantidade medida ou calculada.
Para ilustrar este problema, considere o peso de uma mala de viagens $32
\text{kg}$ exatamente, porém ao medir o peso desta mala em uma balança
analógica verificamos que a leitura do valor é de $31.8 \text{kg}$. Como
poderíamos saber se este erro está dentro da tolerância apresentada pelo
fabricante da balança?

Erro absoluto = valor aproximado - valor exato

Erro relativo = erro absoluto / valor exato

Normas em vetor:

\begin{equation}
 ||x||_p = \bigg( \sum_{i=1}^{n} |x_i|^p \bigg)^{1/p}
\end{equation}

Em computação científica, três normas destacam-se:

\begin{itemize}
 \item norma 1:
\end{itemize}

\begin{equation}
 ||x||_1 = \sum_{i=1}^{n} |x_i|
\end{equation}

\begin{itemize}
 \item norma 2:
\end{itemize}

\begin{equation}
 ||x||_2 = \bigg( \sum_{i=1}^{n} |x_i|^2 \bigg)^{1/2}
\end{equation}

\begin{itemize}
 \item norma $\infty$ ou norma \textit{inf}:
\end{itemize}

\begin{equation}
 ||x||_{\infty} = \text{\textbf{max}} |x_i|
\end{equation}

Para ilustrar o exemplo, considere os seguintes vetores:

\begin{align}
 x &= 
  \begin{bmatrix}
   1   \\
   100 \\
   15  \\
   7   
  \end{bmatrix}
\end{align}

\begin{align}
\hat{x} &= \begin{bmatrix}
  1.1  \\
  98   \\
  14   \\
  5.8   
\end{bmatrix}
\end{align}

Onde $x$ representa o vetor de valores exatos e $\hat{x}$ o vetor de
valores aproximados. O erro absoluto pode ser calculado nas três normas
apresentadas da seguinte forma:

\begin{equation}
 ||x - \hat{x}||_1 =  
 = 
 \sum_{i=1}^{3} |x_i - \hat{x}_i|
 =
\end{equation}

\begin{equation}
 ||x - \hat{x}||_2 =  
 = 
 \bigg( \sum_{i=1}^{3} |x_i - \hat{x}_i|^2 \bigg)^2
 =
\end{equation}

\begin{equation}
 ||x - \hat{x}||_{\infty} =  
 = 
 \text{\textit{max}}_{i=1}^{3} |x_i - \hat{x}_i|
 =
\end{equation}


\typeout{ ****************** End of file erro.tex ****************** }

