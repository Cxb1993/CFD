\typeout{ ====================================================================}
\typeout{ this is file particulas.tex, created at 10-Apr-2015                 }
\typeout{ maintained by Gustavo Rabello dos Anjos                             }
\typeout{ e-mail: gustavo.rabello@gmail.com                                   }
\typeout{ ====================================================================}

\section{PARTÍCULAS}

A equação de Basset-Boussinesq-Oseen (BBO) relaciona a taxa de variação
de quantidade de movimento de uma partícula esférica rígida imersa num
campo de velocidade caracterizado por baixo número de Reynolds. Esta
equação acopla o movimento das partículas (fase dispersa) ao escoamento
do fluido (fase contínua) por meio da transferência de quantidade de
movimento linear, modelando a ação do fluido na partícula. Em processos
reais, as partículas também podem exercer influência sobre o fluido. No
entanto, a concentrações de partículas pode ser suficientemente baixa
tal que o efeito destas forças no fluido pode ser desprezado, de maneira
que o acoplamento entre fluido e partícula se dá apenas no sentido do
fluido para as partículas, constituindo um modelo do tipo ({\it one-way
coupling}). Além disso, problemas como o transporte de sedimentos em
grandes corpos aquáticos, dispersão partículas de poluição na atmosfera,
entre outros, as escalas de comprimento das partículas são muito
pequenas em relação às do escoamento da fase contínua. Nestes casos, o
conceito de partículas pontuais pode ser empregado, segundo o qual a
influência do volume no escoamento pode ser desconsiderada.

% incluir figura ilustrativa de densidade de partículas para elucidar
% acoplamentos one-way, two-way e four-way.

Dada a massa $M$ de uma partícula pontual, com densidade $\rho_p$,
diâmetro $D$ e o campo de velocidade velocidade $\vec{v}_p$
(bidimensional), consideremos agora as seguintes três forças atuando na
partícula: arrasto ({\it drag}, $\vec{F}_D$), sustentação ({\it lift},
$\vec{F}_L$) e força de ``massa acrescida'' ({\it added mass},
$\vec{F}_AM$), definidas da seguinte forma:

arrasto $\vec{F}_D$:
\begin{equation}
\vec{F}_D = \frac{3\rho}{4D}C_D(\vec{u}-\vec{v}_p)\parallel\vec{u}-\vec{v}_p\parallel
\label{eq:arrasto}
\end{equation}

sustentação $\vec{F}_L$:
\begin{equation}
\vec{F}_L = 1.61\sqrt{\mu\rho_p}D^2\parallel\vec{u}-\vec{v}_p\parallel\frac{d\vec{u}}{dy}\left|\frac{du}{dy}\right|^{-0.5}
\label{eq:sustentacao}
\end{equation}

added mass $\vec{F}_{AM}$:
\begin{equation}
\vec{F}_{AM} = \rho\frac{\pi D^3}{12}\frac{d}{dt}(\vec{v}_p-\vec{u})
\label{eq:added_mass}
\end{equation}
%gravity force $\vec{F}_G$:
%\begin{equation}
%\vec{F}_G = M\vec{g}
%\end{equation}
onde o vetor $\vec{u}$ é o campo de velocidade bidimensional do
escoamento (o qual, no caso do canal, é tal que $\vec{u}=u\vec{i}$,
i.e., a componente na direção $y$ é nula). No termo da força de arrasto,
$C_D$ é o coeficiente de arrasto, definido por \cite{PERRY99}

\begin{eqnarray}
	C_D &=& \frac{24}{Re_p}\text{, para }\Re_p< 0,1\nonumber\\
	C_D &=& \frac{24}{Re_p}\left(1+0,14\Re_p^{0,7}\right)\text{, para }0,1\leq\Re_p< 1000\\
	C_D &=& 0,445\text{, para }1000\leq\Re_p< 350000\nonumber
\end{eqnarray}
onde foi introduzido o número de Reynolds da partícula $Re_p$ definido
por

\begin{equation}
	Re_p = \frac{\rho}{\mu}D\parallel\vec{u}-\vec{v}_p\parallel
\end{equation}

Para estas três forças \ref{eq:arrasto}, \ref{eq:sustentacao} e
\ref{eq:added_mass}, a equação BBO é expressa por

\begin{equation}
	M\frac{d\vec{v}_p}{dt} = 
	\frac{3\rho}{4D}C_D(\vec{u}-\vec{v}_p)\parallel\vec{u}-\vec{v}_p\parallel
	+1.61\sqrt{\mu\rho_p}D^2\parallel\vec{u}-\vec{v}_p\parallel\frac{d\vec{u}}{dy}\left|\frac{du}{dy}\right|^{-0.5}
	+\rho\frac{\pi D^3}{12}\frac{d}{dt}(\vec{v}_p-\vec{u})
\end{equation}

cuja solução dentro do intervalo do passo de tempo fornece a velocidade
$\vec{v}_p$ da partícula. O {\it tracking} da partícula pode então ser
efetuado por meio do cálculo da posição com a velocidade $\vec{v}_p$.

Embora não seja o caso do escoamento em canal do qual trata o presente texto, cabe citar ainda a força exercida pelo campo gravitacional na partícula, $\vec{F}_G$, a qual resulta numa força de empuxo sobre a partícula dada por

gravidade $\vec{F}_G$:
\begin{equation}
	\vec{F}_G = (\rho_p-\rho)\pi\frac{D^3}{6}\vec{g}
\end{equation}

Outras forças podem ser incluídas no modelo do movimento da partícula,
como influência de campo magnético, campo elétrico, ou a força de
Basset. Maiores detalhes podem ser obtidos de \citet{crowe2012}.

%\begin{equation}
%	M \frac{du_p}{dt} 
%	=
%	3 \pi \mu d_p (u_f-u_p)
%	-
%	\frac{M}{\rho_p} \nabla p
%	+
%	\frac{M}{2} \frac{d}{dt}(u_f-u_p)
%	+
%	\frac{3}{2} d_p^2 \sqrt{\pi \rho_f \mu}
%	+
%	\sum F_k
%\end{equation}
%
%\noindent onde o primeiro termo do lado direto da equação representa o
%arrasto, o segundo termo representa o gradiente de pressão, o terceiro
%a força virtual, o quarto a força de Basset e o quinto termo representa
%qualquer outra força que deva ser levada em consideração, como por
%exemplo a força de gravidade.
%
%\begin{equation}
%	\sum F = ma = m \frac{D \mathbf{u}}{D t}
%\end{equation}
%
%Em $\sum F$ diversas forças podem ser consideradas no modelo dependendo
%do grau de complexidade e precisão do esquema proposto. Neste seção,
%consideraremos àquelas pertinentes ao desenvolvimento do um modelo
%simplificado, cabendo ressaltar que a consideração de outras forças
%varia de acordo com a aplicação em problemas mais complexos. Para
%maiores informações, ao leitor é sugerido as seguintes referências:
%\cite{crowe2012}, 
%
%\begin{equation}
%	\sum F = F_{lift} + F_{drag} + F_{gravity} + F_{virtual}
%\end{equation}
%
%\noindent onde $F_{gravity}$ representa a força gravitacional definida
%como: 
%
%\begin{equation}
%	dF_{gravity}
%	=
%	\rho_p g dv
%\end{equation}
%
%ao integramos no volume ocupado por uma partícula qualquer, a força
%gravitacional toma a forma de:
%
%\begin{equation}
%	dF_{gravity}
%	=
%	\rho_p \pi \frac{d_p^3}{6} g
%\end{equation}
%
%\noindent onde $dv=\pi d_p^3/6$. $F_{lift}$ representa a força de
%sustentação (``lift'')











\typeout{ ****************** End of file particulas.tex ****************** }

