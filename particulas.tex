\typeout{ ====================================================================}
\typeout{ this is file particulas.tex, created at 10-Apr-2015               }
\typeout{ maintained by Gustavo Rabello dos Anjos                             }
\typeout{ e-mail: gustavo.rabello@gmail.com                                   }
\typeout{ ====================================================================}

\section{PARTÍCULAS}

Em dinâmica dos fluidos, a equação de Basset-Boussinesq-Oseen (BBO)
descreve o movimento e as forças que atuam em partículas pequenas para
escoamentos transiente com baixo número de Reynolds. Sua formulação pode
ser encontrada de diversas formas. 

\begin{equation}
	\frac{\pi}{6} \rho_p d_p^3 \frac{du_p}{dt} 
	=
	3 \pi \mu d_p (u_f-u_p)
	-
	\frac{\pi}{6} d_p^3 \nabla p
	+
	\frac{\pi}{12} \rho_f d_p^3 \frac{d}{dt}(u_f-u_p)
	+
	\frac{3}{2} d_p^2 \sqrt(\pi \rho_f \mu) \cdots
	+
	\sum F_k
\end{equation}

\noindent onde o primeiro termo do lado direto da equação representa o
arrasto, o segundo termo representa o gradiente de pressão, o terceiro
a força virtual, o quarto a força de Basset e o quinto termo representa
qualquer outra força que deva ser levada em consideração, como por
exemplo a força de gravidade.

\begin{equation}
	\sum F = ma = m \frac{D \mathbf{u}}{D t}
\end{equation}

Em $\sum F$ diversas forças podem ser consideradas no modelo dependendo
do grau de complexidade e precisão do esquema proposto. Neste seção,
consideraremos àquelas pertinentes ao desenvolvimento do um modelo
simplificado, cabendo ressaltar que a consideração de outras forças
varia de acordo com a aplicação em problemas mais complexos. Para
maiores informações, ao leitor é sugerido as seguintes referências:
\cite{crowe2012}, 

\begin{equation}
	\sum F = F_{lift} + F_{drag} + F_{gravity} + F_{virtual}
\end{equation}

\noindent onde $F_{gravity}$ representa a força gravitacional definida
como: 

\begin{equation}
	dF_{gravity}
	=
	\rho_p g dv
\end{equation}

ao integramos no volume ocupado por uma partícula qualquer, a força
gravitacional toma a forma de:

\begin{equation}
	dF_{gravity}
	=
	\rho_p \pi \frac{d_p^3}{6} g
\end{equation}

\noindent onde $dv=\pi d_p^3/6$. $F_{lift}$ representa a força de
sustentação (``lift'')











\typeout{ ****************** End of file particulas.tex ****************** }

