
\section{PERFIL TURBULENTO}

Para números de Reynolds acima de um certo limite\footnote{Reynolds observou em experimento os regimes pelos quais um escoamento pode passar e quais os parâmetros influenciavam na transição. Suas conclusões em cima desse trabalho foram publicadas em 1883\cite{REYNOLDS83,DAVIDSON11}}, eventuais perturbações introduzidas ao escoamento podem gerar oscilações cujas amplitudes cresçam monotonicamente, tornando-o instável hidrodinamicamente e convertendo o regime laminar em turbulento. Escoamentos turbulentos são caracterizados por

\begin{itemize}
\item alto grau de mistura
\item riqueza de escalas
\item caos
\end{itemize}

Osborne Reynolds introduziu a abordagem estatística ao estudo de escoamentos turbulentos em 1895 \cite{REYNOLDS95}, segundo a qual o escoamento médio é resolvido. Para escoamentos quase estacionários, médias temporais podem ser usadas \cite{FREIRE02}. Matematicamente, o conceito introduzido por Reynolds consistia na média das equações de Navier-Stokes, cujo resultado são as equações RANS -- Reynolds Averaged Navier-Stokes. Para escoamentos incompressíveis, elas são dadas por

\begin{equation}
  \frac{\D\vec{v}}{\D t}+\vec{v}\cdot\grad\vec{v}=-\frac{1}{\rho}\grad p+\nu\nabla^2\vec{v}-\div\overline{\tilde{\vec{v}}\otimes\tilde{\vec{v}}}
  \label{eq:rans_incompressivel_1}
\end{equation}

Expandindo em coordenadas cartesianas temos

\begin{subequations}
  \begin{eqnarray}
    \frac{\D u}{\D t}+u\frac{\D u}{\D x}+v\frac{\D u}{\D y}+w\frac{\D u}{\D z}&=&-\frac{1}{\rho}\frac{\D p}{\D x}+\nu\left(\frac{\D^2u}{\D x^2}+\frac{\D^2u}{\D y^2}+\frac{\D^2u}{\D z^2}\right)-\left(\frac{\D}{\D x}\overline{\tilde{u}\tilde{u}}+\frac{\D}{\D y}\overline{\tilde{u}\tilde{v}}+\frac{\D}{\D z}\overline{\tilde{u}\tilde{w}}\right)\\
    \frac{\D v}{\D t}+u\frac{\D v}{\D x}+v\frac{\D v}{\D y}+w\frac{\D v}{\D z}&=&-\frac{1}{\rho}\frac{\D p}{\D y}+\nu\left(\frac{\D^2v}{\D x^2}+\frac{\D^2v}{\D y^2}+\frac{\D^2v}{\D z^2}\right)-\left(\frac{\D}{\D x}\overline{\tilde{v}\tilde{u}}+\frac{\D}{\D y}\overline{\tilde{v}\tilde{v}}+\frac{\D}{\D z}\overline{\tilde{v}\tilde{w}}\right)\\
    \frac{\D w}{\D t}+u\frac{\D w}{\D x}+v\frac{\D w}{\D y}+w\frac{\D w}{\D z}&=&-\frac{1}{\rho}\frac{\D p}{\D z}+\nu\left(\frac{\D^2w}{\D x^2}+\frac{\D^2w}{\D y^2}+\frac{\D^2w}{\D z^2}\right)-\left(\frac{\D}{\D x}\overline{\tilde{w}\tilde{u}}+\frac{\D}{\D y}\overline{\tilde{w}\tilde{v}}+\frac{\D}{\D z}\overline{\tilde{w}\tilde{w}}\right)
  \end{eqnarray}
\end{subequations}
É importante notar que o campo de velocidade $\vec{v}$ e suas componentes, bem como o campo de pressão $p$, expressam aqui os respectivos valores médios, enquanto que o campo $\tilde{\vec{v}}$ e suas componentes representam as flutuações em torno dos valores médios, conforme a decomposição de Reynolds.

A dissipação introduzida pelos termos não-lineares de flutuação pode ser interpretada como um campo de tensão adicional atuando no escoamento, representado pelo tensor de tensões turbulentas dado por $\tau_t=-\rho\overline{\tilde{\vec{v}}\otimes\tilde{\vec{v}}}$, cujo divergente é

\begin{equation}
  \div\tau_t = -\rho\div\overline{\tilde{\vec{v}}\otimes\tilde{\vec{v}}}
\end{equation}

A equação \ref{eq:rans_incompressivel_1} pode portanto ser reescrita como

\begin{equation}
  \frac{\D\vec{v}}{\D t}+\vec{v}\cdot\grad\vec{v}=-\frac{1}{\rho}\grad p+\frac{1}{\rho}\div\left(\mu\grad\vec{v}+\tau_t\right)
  \label{eq:rans_incompressivel_2}
\end{equation}

Esta formulação introduz novas variáveis ao modelo, indeterminando o sistema de equações. A estratégia clássica de fechamento do sistema é a aplicação da hipótese de Boussinesq, proposta em 1877\cite{FREIRE02}\footnote{O artigo de Reynolds propondo a decomposição dos campos em parcelas média e flutuante e dando origem ao hoje chamado tensor de Reynolds foi publicado somente em 1895. O que havia como base de conhecimento para Boussinesq fazer essa proposta em 1877?}, segundo a qual os processos de difusão da quantidade de movimento molecular e turbulento são análogos. Matematicamente, isso equivale a

\begin{equation}
  \tau_t = \mu_t\grad\vec{v}
\end{equation}
onde $\mu_t$ é dita viscosidade turbulenta, ou viscosidade de Boussinesq. É mais conveniente agora nos referirmos à viscosidade molecular $\mu_m$. No caso da viscosidade cinética temos portanto $\nu_t$ e $\nu_m$. Para a maioria dos números de Reynolds, $\nu_t$ é algumas ordens de grandeza superior a $\nu_m$, ou seja, as dissipações turbulentas são muito maiores do que as dissipações viscosas. Introduzindo o conceito de viscosidade turbulenta na equação \ref{eq:rans_incompressivel_2}, obtemos

\begin{equation}
  \frac{\D\vec{v}}{\D t}+\vec{v}\cdot\grad\vec{v}=-\frac{1}{\rho}\grad p+\div\left[(\nu_m+\nu_t)\grad\vec{v}\right]
  \label{eq:rans_incompressivel_boussinesq1}
\end{equation}

A viscosidade adicional $\mu_t$ pode ser interpretada como um acréscimo dos efeitos de dissipação ao escoamento, e deve estar associada às características do escoamento.

Para o escoamento desenvolvido no canal, a equação \ref{eq:rans_incompressivel_boussinesq1} se reduz a

\begin{equation}
  \frac{\D u}{\D t}=-\frac{1}{\rho}\frac{\D p}{\D x}+\frac{\D}{\D y}\left[(\nu_m+\nu_t)\frac{\D u}{\D y}\right]
  \label{eq:rans_incompressivel_boussinesq2}
\end{equation}

\subsection{Camada limite turbulenta}

A região do escoamento próxima à parede passa por transições importantes até chegar ao escoamento principal\footnote{A expressão em inglês para o escoamento principal seria ``bulk flow''.}. Essa região é a camada limite do escoamento. É bem aceito que a camada limite possui duas regiões distintas: uma adjacente à parede, na qual os efeitos viscosos predominam -- subcamada viscosa -- e uma seguinte na qual os efeitos turbulentos são mais importantes -- subcamada turbulenta. Cada uma delas possui um perfil de velocidade diferente. Essa composição é conhecida como a estrutura assintótica da camada limite turbulenta.

Na subcamada viscosa, a condição do escoamento pode ser descrita por

\begin{equation}
  \mu\frac{\D^2u}{\D y^2} = 0
\end{equation}

Após integração, chegamos a

\begin{equation}
  u = \frac{Cy}{\mu}
\end{equation}
Tendo em vista que a tensão cisalhante $\tau_w$ na parede é

\begin{equation}
  \tau_w = \mu\frac{\D u}{\D y}
\end{equation}
temos que

\begin{equation}
  u = \frac{\tau_wy}{\mu}
  \label{eq:subcamada_viscosa_dim}
\end{equation}
Finalmente, introduzindo uma velocidade $u_{\tau}$, denominada velocidade de atrito, definida por

\begin{equation}
  u_{\tau} = \sqrt{\frac{\tau_w}{\rho}}
\end{equation}
podemos adimensionalizar a equação \ref{eq:subcamada_viscosa_dim} definindo

\begin{subequations}
  \begin{eqnarray}
    u^+ &=& \frac{u}{u_{\tau}}\\
    y^+ &=& \frac{yu_{\tau}}{\nu}
  \end{eqnarray}  
\end{subequations}
De tal forma que, para a subcamada viscosa, temos

\begin{equation}
  u^+=y^+
\end{equation}

Para a subcamada turbulenta, é necessária uma avaliação das ordens de grandeza dos termos importantes, utilizando o conceito de comprimento de mistura. \citet{FREIRE02} descrevem o desenvolvimento da expressão para a região turbulenta da camada limite. Aqui, vamos nos limitar a dizer que o perfil de velocidade na subcamada turbulenta é expresso por

\begin{equation}
  u^+=\frac{1}{\kappa}\ln y^+ + B
\end{equation}
onde $\kappa$ é a constante de Von Kármán, sendo normalmente $\kappa=0,41$, e $B = 5$ é um valor constante bem aceito para escoamentos em parede, baseado em resultados experimentais. A figura 3.13 de \citet{FREIRE02} apresenta um conjunto de perfis para a camada limite turbulenta obtidos de experimentos diversos.

\subsection{Comprimento de mistura de Prandtl}

A viscosidade turbulenta $\nu_t$ pode ser determinada através de modelo algébrico ou de modelo a uma equação diferencial ou de modelo a duas equações diferenciais (neste se enquadram os modelos $\kappa$-$\epsilon$ e $\kappa$-$\omega$).

O modelo algébrico é baseado no conceito de comprimento de mistura, concebido por Ludwig Prandtl (ver \citet{FREIRE02}, capítulo 3), segundo o qual

\begin{equation}
  \nu_t = l_c^2\left|\frac{\D u}{\D y}\right|
\end{equation}
onde $l_c$ é o comprimento de mistura. Para escoamentos próximos a paredes sólidas (caso do escoamento no canal), 

\begin{subequations}
  \begin{eqnarray}
    l_c&=&\kappa y\text{, para }y\leq\delta\\
    l_c&=&\kappa \delta\text{, para }y>\delta
  \end{eqnarray}
\end{subequations}
onde $\kappa$ é a constante de Von Kármán, e $\delta$ é a espessura da camada limite.

Com essa definição, a variação do comprimento de mistura passa por uma descontinuidade da parede para o interior da camada limite. O comprimento de mistura pode ser calculado ainda com a aplicação de uma função de amortecimento. Normalmente é usada a função de amortecimento de Van Driest. Neste caso,

\begin{equation}
  l_c=D\kappa y\text{, para }y\leq\delta
\end{equation}
com

\begin{equation}
  D = 1-\exp\left(-y\frac{u_{\tau}}{A\nu}\right)
\end{equation}
onde $A = 26$.

Cabe ressaltar que a modelagem do escoamento através da introdução da viscosidade turbulenta é uma das formas de solução do Problema de Fechamento dos modelos RANS e é dito modelo de turbulência de primeira ordem. Existem modelos nos quais as componentes do tensor de Reynolds são decompostas, dando a origem a termos com produtos de três componentes de velocidade, classificados como modelos de turbulência de segunda ordem. Um outro ponto é que, no caso particular do escoamento no canal, considerando que ele possua largura $L$, o comprimento de mistura é dado por

\begin{subequations}
  \begin{eqnarray}
    l_c &=& \kappa y\text{, para }y\leq\delta\\
    l_c &=& \kappa \delta\text{, para }\delta<y<L-\delta\\
    l_c &=& \kappa(L-y)\text{, para }y\geq(L-\delta)
  \end{eqnarray}
\end{subequations}

\subsection{Discretização}

A forma discreta da equação \ref{eq:rans_incompressivel_boussinesq2} é igual à equação \ref{eq:perfil_laminar_discr}, isto é,

\begin{eqnarray}
  \frac{u_i^{n+1}-u_i^n}{\Delta t} = -\frac{1}{\rho}\frac{\D p}{\D x} &+& \alpha\frac{\nu_{i+1/2}(u_{i+1}^{n+1}-u_{i}^{n+1})-\nu_{i-1/2}(u_{i}^{n+1}-u_{i-1}^{n+1})}{\Delta y^2}+\nonumber\\ &+& (1-\alpha)\frac{\nu_{i+1/2}(u_{i+1}^n-u_{i}^n)-\nu_{i-1/2}(u_{i}^n-u_{i-1}^n)}{\Delta y^2}
  \label{eq:perfil_turbulento_discr}
\end{eqnarray}
Porém com $\nu = \nu_m+\nu_t$.

