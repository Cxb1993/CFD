
\section{PERFIL LAMINAR}

\subsection{Regime laminar permanente}

Considere as componentes $x$ e $y$ das equações de Navier Stokes para escoamento incompressível (com viscosidade constante\footnote{Para um escoamento isotérmico, a hipótese de viscosidade constante é muito realista, uma vez que, embora ela dependa também da pressão do fluido (além da temperatura), as variações com a pressão são extremamente baixas.} e sem termo das forças de corpo):

\begin{subequations}
  \begin{eqnarray}
    \frac{\D u}{\D t}+u\frac{\D u}{\D x}+v\frac{\D u}{\D y}&=&-\frac{1}{\rho}\frac{\D p}{\D x}+\nu\left(\frac{\D^2u}{\D x^2}+\frac{\D^2u}{\D y^2}\right)\\
    \frac{\D v}{\D t}+u\frac{\D v}{\D x}+v\frac{\D v}{\D y}&=&-\frac{1}{\rho}\frac{\D p}{\D y}+\nu\left(\frac{\D^2v}{\D x^2}+\frac{\D^2v}{\D y^2}\right)
  \end{eqnarray}
\end{subequations}

\begin{equation}
  \frac{\D u}{\D x} + \frac{\D v}{\D y} = 0
\end{equation}

Para o regime permanente do escoamento, $\frac{\D u}{\D t}=\frac{\D v}{\D t}=0$.

Suponha também que o escoamento seja desenvolvido, isto é,

\begin{equation}
  \frac{\D u}{\D x} = 0
\end{equation}
o que implica em

\begin{equation}
  \frac{\D v}{\D y} = 0
\end{equation}

Esse resultado significa que $v = \text{constante}$ ao longo de qualquer seção do canal. Mais do que isso: por causa das condições de contorno de não deslizamento, $v = 0$, em todo o domínio do escoamento.

Temos portanto

\begin{subequations}
  \begin{eqnarray}
    -\frac{1}{\rho}\frac{\D p}{\D x}+\nu\frac{\D^2u}{\D y^2}&=&0\\
    -\frac{1}{\rho}\frac{\D p}{\D y}&=&0
  \end{eqnarray}
  \label{eq:sistema_laminar}
\end{subequations}

A condição de que a derivada parcial da pressão em relação a $y$ seja nula significa que $p = p(x)$, ou seja, o campo de pressão não depende da coordenada $y$. Este resultado significa que o escoamento incompressível, desenvolvido e estacionário em um canal cujos fluxos na direção $z$ são irrelevantes é modelado por

\begin{equation}
  \frac{\D p}{\D x}=\mu\frac{\D^2u}{\D y^2}
  \label{eq:perfil_laminar_edp}
\end{equation}
Uma vez que a pressão depende apenas de $x$ e a velocidade $u$ depende apenas de $y$, podemos concluir que a pressão depende linearmente de $x$, de maneira que $\D p/\D x$ é constante. Dado um gradiente de pressão, a equação \ref{eq:perfil_laminar_edp} se torna uma EDO homogênea de segunda ordem, cuja solução é o perfil laminar do escoamento com as características citadas, expresso por

\begin{equation}
  u(y) = \frac{1}{2\mu}\frac{\D p}{\D x}y^2+C_1y+C_2
\end{equation}
com

\begin{subequations}
  \begin{eqnarray}
    C_1 &=& \frac{u(L)-u(0)}{L}-\frac{1}{2\mu}\frac{\D p}{\D x}L\\
    C_2 &=& u(0)
  \end{eqnarray}
\end{subequations}
onde $L$ é a largura do canal. Se $u(0)=u(L)=0$, obtemos

\begin{equation}
  u(y) = \frac{1}{2\mu}\frac{\D p}{\D x}\left(y^2-Ly\right)
  \label{eq:perfil_laminar}
\end{equation}

Considere que $u(y=L/2)=U$. Para o perfil dado pela equação \ref{eq:perfil_laminar}, temos portanto que

\begin{equation}
  U = \frac{1}{2\mu}\frac{\D p}{\D x}\left(-\frac{L^2}{4}\right)
\end{equation}
o que resulta em

\begin{equation}
  \frac{\D p}{\D x}=-8\frac{\mu U}{L^2}
\end{equation}

Definindo $\Re=UL\rho/\mu$, obtemos

\begin{equation}
  \frac{\D p}{\D x}=-8\frac{\mu^2\Re}{\rho L^3}
  \label{eq:gradp}
\end{equation}

\subsection{Regime laminar transiente}

O regime transiente do mesmo escoamento pode ser representado por

\begin{equation}
    \frac{\D u}{\D t}=-\frac{1}{\rho}\frac{\D p}{\D x}+\nu\frac{\D^2u}{\D y^2}
    \label{eq:perfil_laminar_transiente}
\end{equation}

Note que agora a equação depende da densidade $\rho$ do fluido. De fato, para o regime transiente, a variação do perfil de velocidade com o tempo deve ser influenciada pela inércia do fluido, convergindo, no entanto, para um valor comum para qualquer valor de $\rho$. Note ainda que a pressão agora é também função do tempo, isto é, $p = p(x,t)$ e seu gradiente não pode mais ser calculado pela expressão \ref{eq:gradp}.

\subsubsection{Discretização}

Para cada nó $i$ da malha, para passo de tempo $\Delta T$ e distância entre nós uniforme igual a $\Delta y$, a discretização da equação \ref{eq:perfil_laminar_transiente} em diferenças finitas centradas é expressa por

\begin{eqnarray}
  \frac{u_i^{n+1}-u_i^n}{\Delta t} = -\frac{1}{\rho}\frac{\D p}{\D x} &+& \alpha\frac{\nu_{i+1/2}(u_{i+1}^{n+1}-u_{i}^{n+1})-\nu_{i-1/2}(u_{i}^{n+1}-u_{i-1}^{n+1})}{\Delta y^2}+\nonumber\\ &+& (1-\alpha)\frac{\nu_{i+1/2}(u_{i+1}^n-u_{i}^n)-\nu_{i-1/2}(u_{i}^n-u_{i-1}^n)}{\Delta y^2}
  \label{eq:perfil_laminar_discr}
\end{eqnarray}
com $0\leq\alpha\leq 1$, e com $\nu_{i+1/2} = (\nu_{i}+\nu_{i+1})/2$ e $\nu_{i-1/2} = (\nu_{i-1}+\nu_{i})/2$.

